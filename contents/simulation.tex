% !TEX root = ../main.tex
\chapter{建模与仿真}

\section{求解过程的简述}
根据第二章中推导的反应过程,仿真的过程主要分为以下三步:

1)在COMSOL Multiphysics中,对通电时发生的电化学反应进行建模,得到阴极极板表面的平均$\ce{OH-}$浓度随
时间的变化曲线。

2)利用第1步得到的浓度变化随时间的变化曲线,在MATLAB中计算$\ce{Zr^4+}$的水解程度随时间的变化关系,
再根据$\ce{Zr^4+}$消耗与DNA释放这一正比关系,得到DNA的释放曲线。

3)利用第2步得到的数据,在COMSOL Multiphysics中,对DNA在电场力以及浓度梯度力作用下的扩散、迁移过程进行建模仿真。
\section{反应容器物理模型}
本课题组使用的反应容器为一个长度为$2cm$,直径为$1cm$的圆柱体。作为电极的金薄膜的规格为$20mm×6mm×1mm$。

如图~\ref{fig:container} 所示,根据上述数据,我们可以在COMSOL Multiphysics中完成对反应溶液的物理模型的建立。
由于电极材料为金,不参与反应,内部也考虑为等势体,所以我们只用考虑在溶液体系内发生的反应,在建模时只用对溶液
进行建模。
\begin{figure}[H]
    \centering
    \includegraphics[scale=0.6]{container.jpg}\\
    1)阳极(Anode) 2)阴极(Cathode) 3)电解液(Electrolyte)
    \caption{反应容器物理模型}
    \label{fig:container}
\end{figure}

\section{电化学反应仿真}
\subsection{计算流程}
电解质中每一种离子的通量通过式~\ref{equation:Nernst_Planck}中的Nernst-Planck方程计算得到。再加上
式~\ref{eqation:continuity}连续性方程与式~\ref{equation:neutrality}电中性条件的约束。可以得到离子的
运动情况。
阳极和阴极边界条件通过式~\ref{equation:Butler_Volmer}Butler-Volmer方程给定。电化学反应的过程为:
\begin{equation}
    \mbox{阳极}\quad \ce{2 Cl- - 2e- -> Cl2(g)}
\end{equation}
\begin{equation}
    \mbox{阴极} \quad \ce{2 H2O + 2 e- -> H2(g) + 2OH- }
\end{equation}

根据以上理论我们在COMSOL Multiphysics中进行如下操作:

1)在COMSOL Multiphysics中新建一个一个项目,选择三维模型,添加电化学物理场,
选择考虑电解质迁移影响的“三次分布”电流模型。

2)在几何选项中添加我们上一步完成的物理模型,并定义正负极板电势,以及极板处所发生的的化学反应的方程式与
平衡电位\footnote{使用标准电极电位值},电极动力选表达式选择Butler-Volmer方程。

3)定义溶液中各物质的浓度、电荷数、扩散系数以及溶液电导率等参数。

4)选择细化网格,调整瞬态求解器的时间范围与时间步,开始计算,得到仿真结果。

\subsection{结果与讨论}
图~\ref{fig:cOH1}显示了运行一段时间后,反应容器内$\ce{OH-}$离子的浓度等值面变化。
可以看到阴极处反应产生了大量的$\ce{OH-}$离子,
并在电场力与浓度梯度力的作用下,向溶液体系中扩散。
\begin{figure}[ht]
    \centering
    \subcaptionbox{运行5$s$时$c_{\ce{OH-}}$等值面}% 
                    [6.4cm]{\includegraphics[height=5cm]{cOH等势面5s.png}}
    \hspace{1cm}
    \subcaptionbox{运行15$s$时$c_{\ce{OH-}}$等值面}%
                    %
                    [6.4cm]{\includegraphics[height=5cm]{cOH等势面15s.png}}
    \centering
    \subcaptionbox{运行30$s$时$c_{\ce{OH-}}$等值面}%
                    %
                    [6.4cm]{\includegraphics[height=5cm]{cOH等势面30s.png}}
    \hspace{1cm}
    \subcaptionbox{运行50$s$时$c_{\ce{OH-}}$等值面}%
                    %
                    [6.4cm]{\includegraphics[height=5cm]{cOH等势面50s.png}}
    \caption{$c_{\ce{OH-}}$等值面与运行时间的关系}
    \label{fig:cOH1}
\end{figure}

图~\ref{fig:cOH_distribution}显示,在运行5$s$后,极板上的$c_{\ce{OH-}}$分布基本均匀,
可以近似认为极板上$c_{\ce{OH-}}$处处相等。
\begin{figure}[H]
    \centering
    \includegraphics[width=12cm]{cOH分布.png}
    \caption{运行5s时,极板所在平面$c_{\ce{OH-}}$浓度的分布情况}
    \label{fig:cOH_distribution}
\end{figure}

如图~\ref{fig:cOH1}所示,在运行的前10$s$,由于电信号的激励,电化学反应不断发生,
阴极表面的$c_{\ce{OH-}}$迅速增加,但由于电荷堆积,行程双电层,抑制反应发生,所以
$c_{\ce{OH-}}$的增长速度在变慢。
运行10$s$后,施加的电势被去除,电化学反应停止,溶液中的${\ce{OH-}}$离子主要做自由扩散,
浓度逐渐降低,浓度曲线近似指数函数。
\begin{figure}[H]
    \centering
    \includegraphics[width=12cm]{cOH曲线.png}
    \caption{极板表面平均$c_{\ce{OH-}}$浓度与时间的关系}
    \label{fig:cOH_t}
\end{figure}

\section{Zr$^{4+}$水解反应仿真}
\subsection{计算流程}
根据式~\ref{rate_equation} 化学反应速率方程,我们知道化学反应速率与反应速率常数$k$以及反应物浓度有关。

松井光二教授与大貝理治教授长期从事金属氧化物的研究,这里我们引用\parencite{40005382310,doi:10.1111/j.1151-2916.2002.tb00131.x}
中,对于$\ce{Zr^4+}$水解时
反应速率常数的研究数据,通过插值法获得我们实验条件下的反应速率常数$k$。计算步骤如下:

1)将电化学反应中,极板表面平均$c_{\ce{OH-}}$浓度与时间的关系的数据导出至txt文档中,采样周期为10$ms$。

2)由于生成的$\ce{OH-}$离子的数量级
远高于固定的$\ce{Zr^4+}$离子,所以我们假设在反应过程中$\ce{Zr^4+}$消耗不会引起$\ce{OH-}$浓度的变化,故$\ce{OH-}$的浓度数据
可以直接代入上一步仿真得到的极板表面平均$c_{\ce{OH-}}$浓度与时间的关系曲线。
在MATLAB中调用浓度时间序列的数据,以10$ms$为采样周期,离散的计算$\ce{Zr^4+}$的消耗曲线。即根据此时$\ce{OH-}$的浓度,计算出
水解反应的速率$v$,然后假定在10$ms$周期内速率为常数,计算这10$ms$内消耗的$\ce{Zr^4+}$。

3)将每一个10$ms$周期内消耗的$\ce{Zr^4+}$的量做累加,得到$\ce{Zr^4+}$的消耗曲线。
%代码见附录
\subsection{结果与讨论}

图~\ref{1stZr}显示了,在第一次激励过程中($0<t<60s$),$\ce{Zr^4+}$随时间而变化的消耗曲线。随着电极表面的$\ce{OH-}$浓度增加,
反应速率将增加,10$s$后,电化学反应停止,电极表面的$\ce{OH-}$浓度因为自由扩散而降低,水解速率也降低。最终可以得到在第一个60$s$的
反应过程中,总共消耗了约2.15层$\ce{Zr^4+}$。
\begin{figure}[H]
    \centering
    \includegraphics[width=10cm]{Zr4+消耗曲线.png}
    \caption{第一次激励过程中$\ce{Zr^4+}$的消耗曲线}
    \label{1stZr}
\end{figure}

改变反应时间,代入该时间段内的$\ce{OH-}$浓度与$\ce{Zr^4+}$浓度数据,重复上述操作,可以得到激励次数与本次激励内$\ce{Zr^4+}$的消耗量
的序列,如图~\ref{Zrseq}所示。随着反应的进行,$\ce{ZrO2}$溶解的逆反应强度会加剧,$\ce{Zr^4+}$的消耗会导致其浓度降低,最终水解速率会
越来越慢,直至反应达到平衡。在课题组实验中,经过5次电压激励后,DNA的浓度基本不再发生改变,这与仿真结果中$\ce{Zr^4+}$的消耗曲线相吻合。
\begin{figure}[H]
    \centering
    \includegraphics[height=6cm]{每次激励消耗.png}
    \caption{第一次激励过程中$\ce{Zr^4+}$的消耗曲线}
    \label{Zrseq}
\end{figure}

\section{DNA扩散过程仿真}
\subsection{计算流程}
根据式~\ref{equation:Nernst_Planck} Nernst Planck普朗克方程,可以计算出了流体介质中带电化学物质的运动情况。实验所用的DNA为碱基长度
在$20$~$100$左右的单链DNA(ssDNA),在\parencite{Stellwagen2003}一文中可以得到,
在这一长度的DNA分子的扩散系数大约是$D_i=1.52\cdot 10^6cm^2 s^{-1}$。根据~\ref{Nernst-Einstein} Nernst Einstein关系,可以得到
扩散系数$D_i$与扩散迁移率$m_{i,abs}$的关系。再加上式~\ref{eqation:continuity}连续性方程与式~\ref{equation:neutrality}电中性条件的约束。
可以得到DNA分子在电场力与浓度梯度力作用下的运动情况。主要操作如下:

1)在COMSOL Multiphysics中加入“电泳输送”物理场。输入溶液电导率,电泳物质扩散系数等参数。

2)我们将正极的电压改为实验中施加的周期信号。阴极板处,我们放置一个ssDNA的源,将MATLAB中计算得到的DNA释放速度曲线作为流入源的通量曲线代入。

3)配置网格以及求解器等参数,进行仿真运算。

\subsection{结果与讨论}
图~\ref{fig:cDNA1}显示了DNA浓度等值面与运行时间的关系。在第1$s$、6$s$与12$s$,DNA不断被释放出来,可以看到极板的正面浓度不断增加,
并且在电场力的作用下,DNA迅速的向正极迁移。25$s$与50$s$这两张图显示的是,在去除电势之后,$\ce{OH-}$浓度恢复到一个较低的水平,
不再释放新的DNA到溶液中,此时DNA主要做自由扩散运动,浓度梯度越小时,扩散速度越慢。第65$s$显示的是,在第二次施加电势之后,新的DNA
被释放到溶液环境中,重复上述过程。

\begin{figure}[H]
    \centering
    \subcaptionbox{运行1$s$时DNA浓度等值面}% 
                    [6.4cm]{\includegraphics[height=5cm]{cDNA1s.png}}
    \hspace{1cm}
    \subcaptionbox{运行6$s$时DNA浓度等值面}%
                    %
                    [6.4cm]{\includegraphics[height=5cm]{cDNA6s.png}}
    \centering
    \subcaptionbox{运行12$s$时DNA浓度等值面}%
                    %
                    [6.4cm]{\includegraphics[height=5cm]{cDNA12s.png}}
    \hspace{1cm}
    \subcaptionbox{运行25$s$时DNA浓度等值面}%
                    %
                    [6.4cm]{\includegraphics[height=5cm]{cDNA25s.png}}
    \centering
    \subcaptionbox{运行50$s$时DNA浓度等值面}%
                    %
                    [6.4cm]{\includegraphics[height=5cm]{cDNA50s.png}}
    \hspace{1cm}
    \subcaptionbox{运行65$s$时DNA浓度等值面}%
                    %
                    [6.4cm]{\includegraphics[height=5cm]{cDNA65s.png}}
    \caption{DNA浓度等值面与运行时间的关系}
    \label{fig:cDNA1}
\end{figure}

我们定义阴极一侧的容器底面为$xOy$平面,其圆心为原点,我们在坐标$(0,9mm,0)$处放置一个
浓度探针作为我们的接收机所在的位置,可以看到该探针所在的位置的DNA浓度随时间的变化曲线
如图~\ref{fig:cDNAatP}所示。这个浓度曲线很好的反应了电场力作用与浓度梯度力作用对DNA在
溶液中的运动。
\begin{figure}[H]
    \centering
    \includegraphics[width=10cm]{cDNA.png}
    \caption{探针处DNA浓度}
    \label{fig:cDNAatP}
\end{figure}

根据式~\ref{fick's 2nd} 菲克第二定律,我们可以求出考虑一维方向上自由扩散的情况下,
距离物质源距离为$x$的位置的该物质浓度曲线$\varphi(x,t)$随时间变化的函数:

\begin{equation}
    \varphi(x,t)= \frac{1}{\sqrt {4 \pi Dt}} exp(-\frac{x^2}{4Dt})
\end{equation}

根据上式,我们假设一种物质$D=0.05m^2\cdot s^{-1}$,在物质源为$x=1m$的位置观测其
浓度随时间变化的曲线,我们称之为信道冲击响应曲线,
简称CIR\footnote{channel impulse response}曲线,
如图~\ref{自由扩散示意图}所示。
\begin{figure}[H]
    \centering
    \includegraphics[width=13cm]{自由扩散示意.jpg}
    \caption{自由扩散示意图}
    \label{自由扩散示意图}
\end{figure}

\parencite{8405569}一文中描述了在电场力作用下的CIR曲线,如图~\ref{电迁移}所示,
横轴表示时间,纵轴表示CIR幅度,即被测物质浓度,
红色曲线为电场力作用下的CIR曲线,与之对比的蓝色曲线是自由扩散时的CIR曲线。
\begin{figure}[H]
    \centering
    \includegraphics[width=8cm]{电迁移.png}
    \caption{电迁移与自由扩散下浓度变化曲线}
    \label{电迁移}
\end{figure}

将自有扩散与电迁移的效果相叠加后,再与图~\ref{fig:cDNAatP}对比,
可以发现,对DNA扩散过程进行的仿真,很好的还原了DNA分子在电场力与浓度梯度力共同作用下
的扩散情况。将该探针处DNA浓度的变化曲线与课题组所做的实验结果相比照,都表现了DNA浓度该变量会
随释放次数的增加而减少这一现象。
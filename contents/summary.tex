% !TEX root = ../main.tex

\begin{summary}
这里是全文总结内容。
近年来,分子通信研究高速发展,诞生了很多新理论与新技术。
一种基于DNA/$\ce{Zr^{4+}}$层层自组装结构在电信号控制下分解的技术
,实现了DNA的可控释放。本文使用COMSOL Multiphysics与MATLAB软件,
在课题组实验数据基础上,综合电化学反应原理、化学动力学原理、扩散原理
等对DNA的受控释放过程进行了建模与仿真,重点对激励信号与接收机位置
进行了研究,分析和总结了影响该系统通信速度、抗干扰能力、总发送量
这三个因素的影响。本文主要研究结果如下:

1)通过对分子通信的学习,了解了目前分子的发展现状:目前分子通信的系统领域已经有多种概念模型和基本框架被建立,
但在如何提高纳米机器的性能水平,如何更有效地传输信号分子,如何提升通信距离以及抗干扰能力这三个方面,
仍有很多挑战等待科研人员,

2)通过对有关化学理论的查阅与理解,本文提出了一种DNA/$\ce{Zr^{4+}}$
层层自组装结构的
分解的原理。在电信号的激励下,阴极处会发生式~\ref{equation:cathode_reaction}
所描述的化学反应,这使得阴极处$\ce{OH-}$浓度增加,促使$\ce{Zr^4+}$
发生式~\ref{Zr4+水解}描述的水解反应,$\ce{Zr^4+}$的消耗导致了
DNA/$\ce{Zr^{4+}}$层层自组装结构的分解,被释放到溶液中的DNA在
浓度梯度力与电场力的共同作用下,在溶液中扩散。

3)在COMSOL Multiphysics与MATLAB上对上述反应过程进行数值仿真,
在引用课题组实验的模型、参数的情况下结果较好的还原了实验数据。
验证了电极反应、水解反应、电泳过程共同作用的仿真可以适用于对
该分子通信系统的研究。

4)不考虑电解热效应的情况下,激励电压越高,系统的抗信道间干扰能力越强。
激励时长越短,系统的抗码间干扰能力越强。
激励信号应有持续时间较短、激励电压较高的特性,
高电压的脉冲函数是一个理想的信号源。

5)距离阴极越近的位置,DNA浓度达到峰值的时间越短,峰值强度越大。
系统通信距离越短,通信效果越好。
短时长的信号的激励下,相同位置处,DNA浓度峰值会更早到达,
峰值相对于整条曲线也会更突出。
在距离阴极0.5mm的地方,
我们可以达到0.909Hz的最大通信频率;在距离阴极4.5mm处,我们可以达到0.2778Hz的最大通信频率,
而在距离阴极9.5mm处,最大通信频率只有0.05348Hz。
数据说明了,基于自由扩散的通信系统,不适合远距离通信。接收机放置在距发送机5$mm$这一范围内时,
能达到一个较快的通信速度,而且抗干扰能力强。在超过7$mm$这一范围后,通讯的间隔必须增加到数十秒,
以抑制码间干扰,而且其抗信道间干扰能力也会下降。

6)综合考虑平衡常数,在$\ce{OH-}$存在时,$\ce{Zr^4+}$水解反应
都会发生,由于反应速度与$c_{\ce{OH-}}$的4次方成正比,所以低浓度时,反应速度极慢,在所观测的
时间尺度下,可以认为不施加激励电压时反应不发生。
电化学反应的电压阈值应为2.17$V$,但实际条件下,
由于电极阻抗、溶液电导率等因素的影响,可能需要更大的电压来启动此反应。
通过对激励电压与时长的控制,一个DNA/$\ce{Zr^4+}$单层可以被多次释放,
在激励电压为5$V$激励时长为1$s$的信号作用下每一层$\ce{Zr^4+}$/DNA单层最多可以释放6次。


\end{summary}

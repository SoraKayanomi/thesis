% !TEX root = ../main.tex

\begin{summary}
针对分子通信领域缺少微观实验平台的瓶颈问题,
本课题组提出了以DNA为信息分子的微观分子通信实验平台
 ,并初步完成了其中关键部件的DNA信息分子的发送器与接收器的设计与实现
。然而,针对DNA分子通信,领域内并没有相关的理论研究。但是,
对实验平台的深入理解,离不开理论研究的指导。针对这个问题,
本毕业设计立足于课题组的现有的DNA分子通信实验平台,对其中DNA信息分子发送器部分开展理论研究。\\

首先,我们融合化学、物理学科的知识,利用COMSOL Multiphysics与MATLAB软件,基于电化学反应原理、化学动力学原理、粒子自由扩散与带电粒子在电场辅助下自由扩散的原理,获得了DNA信息分子发送器的三维理论模型。其次,我们利用获得的DNA信息分子发送器模型,研究了关键参数对DNA发送过程的影响,包括:激励信号与接收机位置等,并分析了关键系统参数对信息传播速度、抗干扰能力、总发送量这三个性能的影响。通过本工作的研究,课题组深入理解了DNA信息分子发送器的特征,指明了优化DNA信息分子发送器的具体方向。对进一步推进本课题组的微观分子通信实验平台的研究具有重要意义。\\

本工作的主要研究结果包含如下几个方面:

1)通过学习,了解了分子通信领域的发展现状与分子通信的基本原理。 

2)通过调研与阅读多学科文献,学习、归纳并总结了DNA信息分子发送器中关键的多层DNA结构的组装过程
、理化性质以及DNA分子的扩散运动运动过程。通过自己的理解与调研
,详细分析与讨论了DNA信息分子分解与释放过程中的电化学反应、水解反应、
电迁移及扩散过程,并最终提出DNA/$\ce{Zr^4+}$多层结构的分解与释放的原理。并
将上述多个过程表达为对应的理论方程。

3)利用两个软件,COMSOL Multiphysics 与 MATLAB,对上述DNA信息分子发送过
程进行数值仿真,仿真结果与课题组的实验数据符合度好,证明了所提出的模型的准确性与有效性。
也验证了利用COMSOL Multiphysics 与 MATLAB对电极反应、水解反应、
电泳等多个过程共同作用的仿真可以用于对DNA分子通信系统的建模。

4)对DNA信息分子发送器的系统特征进行研究,深入理解了DNA信息分子发送器的
特征,指明了优化DNA信息分子发送器的具体方向。关键结论包括:

$\bullet$讨论了激励电压信号的强度与持续时间对DNA信息分子发送过程的影响。不考虑电解热效应的情况下,
激励电压越高,系统抗码间串扰的能力越强。激励时长越短,系统抗码间串扰的能力越强。激励信号需要具备持续时间较短
、激励电压较高的特性,高电压的脉冲函数是一个理想的信号源。

$\bullet$讨论了接收机位置对系统性能的影响,获得信道中不同位置初的DNA浓度变化曲线并研究了DNA信息分子发送器
的阈值。接收机距离阴极越近,DNA浓度达到峰值的时间越短,峰值强度越大。接收机距离越短,信号传输效果越好。
短时长的信号的激励下,相同位置处,DNA浓度峰值会更早到达,峰值相对于整条曲线也会更突出。
在距离阴极 0.5$mm$的地方,我们可以达到54.5$bit/min$的最大通信速率;
在距离阴极 4.5$mm$处,我们可以达到16.7$bit/min$的最大通信频率,而在距离阴极9.5$mm$处,最大通信频率只有3.2$bit/min$。
数据说明了,基于自由扩散的通信系统,
不适合远距离通信。接收机放置在距发送机5$mm$以内这一范围内时,能达到一个较快的传输速度,
而且抗干扰能力强。在超过7$mm$这一范围后,
信号间隔必须增加到数十秒,以抑制码间串扰,而且其抗码间串扰的能力能力也会下降。

$\bullet$综合考虑平衡常数,在$\ce{OH-}$存在时,$\ce{Zr^4+}$水解反应都会发生,
由于反应速度与$c_{\ce{OH-}}$的4次方成正比,所以低$\ce{OH-}$浓度时,
反应速度极慢,在所观测的时间尺度下,可以认为不施加激励电压时反应不发生。
电化学反应的电压阈值应为2.17$V$。但实际条件下,由于电极阻抗、溶液电导率等因素的影响,
可能需要更大的电压来启动此反应。通过对激励电压与时长的控制,一个DNA/$\ce{Zr^4+}$单层可以被多次释放,
在激励电压为5$V$,激励时长为1$s$的信号作用下,每一层DNA/$\ce{Zr^4+}$单层最多可以释放6次。


\end{summary}

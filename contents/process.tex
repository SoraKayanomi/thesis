% !TEX root = ../main.tex
\chapter{DNA分子的释放过程简述}

\section{DNA/Zr$^{4+}$层层自组装结构}
\subsection{组装过程}
——用到的组装方法
\subsection{理化性质}
——介绍该结构的物理与化学性质

\section{层层自组装结构受控分解的原理}
\subsection{电信号激励下的电化学反应}
——通过电极电位表等客观条件,推断出通电后发生的电极反应
\subsection{Zr$^{4+}$水解过程}
——介绍Zr$^{4+}$水解相关的内容
\subsection{电场影响下DNA分子运动}
——介绍在浓度梯度力与电场力共同作用下DNA的行为
\subsection{反应过程总述}
——总结反应过程,插入反应过程图等

\section{基本控制方程}
\subsection{扩散定律}
菲克定律由阿道夫·菲克于十九世纪提出,对扩散给出了最简单的描述:
\begin{itemize}[leftmargin=60pt]
    \item [1)]由扩散引起的摩尔通量与浓度梯度成正比。
    \item [2)]空间上某一点的浓度变化率与浓度的空间二阶导数成正比。
\end{itemize}

菲克第一定律可以表示为:
\begin{equation}
    \mathbf{N_i}=-D_i\nabla{c_i}
\end{equation}

对于物质$i$,$\mathbf{N_i}$是摩尔通量($mol\cdot{m^{-2}\cdot{s^{-1}}}$),$D_i$是扩散系数($m^2\cdot{s^{-1}}$),$c_i$是浓度($mol\cdot{m^{-3}}$)。
根据连续性方程:
\begin{equation}
    \frac{\partial c_i}{\partial t}+\nabla\cdot{N_i}=0
\end{equation}

可以推导出菲克第二定律:
\begin{equation}
    \frac{\partial c_i}{\partial t}=D_i\nabla^2{c_i}
\end{equation}

在稀溶液中,我们假设$D_i$是一个常数。它描述了自由扩散时,浓度随时间的变化关系。\cite{Sakaguchi2018} 
\subsection{Nernst-Einstein 关系}
考虑外力作用于扩散粒子:
\begin{equation}
    \mathbf{v_d}=m_{abs}\mathbf{F}
\end{equation}

在这一本构关系中,$\mathbf{v_d}$表示漂移速度($m\cdot{s^{-1}}$),
$m_{abs}$是绝对迁移率($N\cdot{s\cdot{m^{-1}}}$),$F$是作用力($N$)。
考虑浓度梯度力同时作用于粒子,粒子的总通量$\mathbf{N_i}$可以表达为:
\begin{equation}
    \mathbf{N_i}=-D_i\nabla{c_i}+m_{i,abs}c_i\mathbf{F}
\end{equation}

当总通量为$0$时,扩散作用于迁移作用引起的通量大小相等且方向相反:
\begin{equation}
    \nabla{c_i}=\frac{m_{i,abs}}{D_i}c_i\mathbf{F}
\end{equation}

各个方向上的净通量为$0$时,可以假设扩散与迁移达到平衡。此时的浓度分部可以用Boltzmann方程来描述:
\begin{equation}
    c_i=c_{i,0}exp(-\frac{U}{kT})
\end{equation}

其中,$c_{i,0}$表示势能为零时的浓度($mol\cdot{m^{-3}}$),$U$表示分子的势能($J$),
$k$表示Boltzmann常数($J\cdot{K^{-1}}$),T是绝对温度(K)。此时该浓度场的梯度为:
\begin{equation}
    \nabla{c_i}=-c_{i,0}exp(-\frac{U}{kT})\frac{1}{kT}\nabla{U}
\end{equation}

根据定义,力是势能的负梯度:
\begin{equation}
    \mathbf{F}=-\nabla{U}
\end{equation}

结合等式(2.7)与(2.9):
\begin{equation}
    \frac{m_{i,abs}}{D_i}c_i\mathbf{F}=\frac{1}{kT}c_i\mathbf{F}
\end{equation}

化简后得到:
\begin{equation}
    m_{i,abs}=\frac{D_i}{kT}
\end{equation}

该等式描述了带点粒子的扩散系数$D_i$与扩散迁移率$m_{i,abs}$的关系,被称为Nernst-Einstein关系。\cite{Mehrer2007,CONWAY1972250}
\subsection{Nernst-Planck 方程}
在电场作用下,带电粒子受到电场力$\mathbf{F}$:
\begin{equation}
    \mathbf{F}=-z_ie_0\nabla\phi
\end{equation}

其中,$z_i$表示粒子的电荷数,$e_0$是电子的基本电荷,$\phi$是电势($V$),$−\nabla{\phi}$ 表示电场。
我们定义$u_i=e_0m_{abs}$为该粒子的电化学迁移率,此时,漂移速度可由下式求得:
\begin{equation}
   \mathbf{v_d}=-z_iu_i\nabla\phi
\end{equation}

由此得到的离子$i$的迁移通量是漂移速度$v_d$和离子浓度$c_i$的乘积,此通量的贡献称为离子迁移或电迁移:
\begin{equation}
    \mathbf{N_{i,migr}}=-z_iu_ic_i\nabla\phi
\end{equation}

在一般的稀释电解质中,通量贡献可能有以下三种来源:扩散、迁移和对流:
\begin{equation}
    N_i=-\overbrace{D_i\nabla{c_i}}^{Diff}-\overbrace{z_iu_ic_i\nabla\phi}^{Migr}+\overbrace{c_i\mathbf{u}}^{Conv}
\end{equation}

其中$\mathbf{u}$是电解质速度($m\cdot{s^{-1}}$)。由Nernst-Einstein关系我们得到电化学迁移率$u_i$与扩散率$D_i$的关系。
代入到式(2.2)的连续性方程:
\begin{equation}
    \frac{\partial c_i}{\partial t}=\nabla{(D_i\nabla{c_i}+\frac{D_iz_ie_0}{kT}c_i\nabla\phi+c_i\mathbf{u})}
\end{equation}

该等式描述了流体介质中带电化学物质的运动情况,被称为Nernst–Planck方程。\cite{Mehrer2007}
\subsection{离子迁移}
电解质中的电流密度$\mathbf{i}$可以通过该电解质中所有离子的贡献之和求得:
\begin{equation}
    \mathbf{i}=F\sum_i{z_i\mathbf{N_i}}
\end{equation}

式中,$F$为法拉第常数。将式(2.)的通量代入到该方程,可以得到:
\begin{equation}
    \mathbf{i}=F\sum_i(-z_iD_i\nabla{c_i}-z_i^2u_ic_i\nabla\phi)+F\sum_iz_ic_i
\end{equation}

在大多数电化学反应中,除双电层区域外,都可以假设电解质呈电中性:
\begin{equation}
    F\sum_iz_ic_i=0
\end{equation}

对流使得浓度保持均匀的分布,因此,在靠近电极的区域外,电解槽中的任何位置都具有恒定的电导率$\kappa$,电流密度表达式变为:
\begin{equation}
    \mathbf{i}=-\overbrace{F\sum_i(z_i^2u_ic_i)}^{Conductivity,\kappa}\nabla\phi
\end{equation}

这个方程表明,组成恒定的电解质中的电流完全由迁移产生。电流遵循欧姆定律,电导率由电解质中每个组成离子的迁移贡献总和决定。\cite{Smedley1980}
\subsection{化学反应速率方程}
——反应物浓度影响反应速度
\subsection{Butler-Volmer 方程}
——电极电势与电化学反应速率的关系


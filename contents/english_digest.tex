% !TEX root = ../main.tex

\begin{digest}
  Molecular communication is an important communication scheme to establish a nano-machine network, which can solve the current problem that the limited capacity of a single nanomachine makes it difficult to carry out any concrete applications. The nano-network built by molecular communication technology can enable the entire nano-machine group to perform more complex tasks in a wider range, which is beneficial to the development of nano-machine applications, such as micro-surgery through the cooperation of a large number of nano-machines or drug delivery inside the human body. Therefore, molecular communication is a promising research direction and has received extensive attention.\\

  At present, the main research work in the field of molecular communication is concentrated on theory. So far, there is no complete microscopic molecular communication experimental platform, which seriously limits the development of the field of molecular communication and becomes a bottleneck in this field. Aiming at the bottleneck in this field, our group proposed a micro-molecular communication experiment platform using DNA as the information molecule, and completed the transmitter and receiver of DNA information molecules, which are the key components of the experiment platform. However, there is no relevant theoretical research on DNA molecular communication in this field. However, a thorough understanding of the experimental platform is inseparable from the guidance of theoretical research.\\
  
  To solve the above problems, this work is based on the existing DNA molecular communication experimental platform of our group and carries out theoretical research on its transmitter part. The transmitter of DNA information molecules that we implemented is a conversion interface from macroscopic electrical signals to microscopic DNA signals. In this contribution we illustrate a simple layer-by-layer assembly strategy to immobilize DNA on the surface of gold thin film and to trigger the selective release of DNA upon the exclusive control ofexternal electric potential. The assembly of DNA-containing multilayer films is driven by coordination/electrostatic interactions between inorganic zirconium ion (Zr4+) and phosphate groups in the backbone of the DNA chain. Its main principle is to use zirconium ion Zr4+ as the adhesive glue, and use the layer-by-layer self-assembly technology to fix the multilayer DNA/Zr4+ structure on the surface of the gold film. The decomposition and controllable release of multilayer DNA structures are performed through electrochemical reactions. This work carries out complete mathematical modeling on the above DNA decomposition and release process. Specifically, we use the electrochemical reaction model, chemical kinetic model and particle free diffusion model in physics to study the overall process of decomposition, release and propagation of the DNA multilayer structure. Using two softwares, COMSOL Multiphysics and MATLAB, this process was studied and a three-dimensional model of DNA decomposition and release process was established. Using the obtained model, the influence of key parameters on the process of DNA decomposition and release was studied, and then the characteristics of the DNA information molecular transmitter were deeply understood, and the specific direction of optimizing the DNA information molecular transmitter was pointed out. It is of great significance to further promote the research of the micro-molecular communication experiment platform in our group.\\
  
  The main work of this work includes the following aspects:
  
  1) A brief introduction to the basic principles and development status of molecular communication. The basic principles of the micro-molecular communication experiment platform of our group were introduced, and the research content and significance of the graduation project work were clarified. The application example of the key software COMSOL Multiphysics of this work in the field of molecular communication is introduced.

  2) The assembly process and physical and chemical properties of the key multi-layer DNA structure in the DNA information molecular transmitter are introduced. The electrochemical reaction, hydrolysis reaction, electromigration and diffusion process during the decomposition and release of DNA information molecules are analyzed and discussed in detail. The theoretical equations expressing the above process and particle motion process are introduced and explained in detail.
  
  3) Using COMSOL Multiphysics, three-dimensional modeling and simulation were carried out on the electrochemical reaction, hydrolysis reaction, electromigration and diffusion process of DNA information molecule decomposition, release and propagation. The obtained DNA concentration curve data from our established model agrees with the previous experimental data. Thus correctness of the obtained model of our DNA information molecular transmitter is validated.
  
  4) Study the system characteristics of DNA information molecular transmitter. The influence of the intensity and duration of the excitation voltage signal on the transmission process of DNA information molecules is discussed, and the DNA concentration change curves at different positions in the channel are obtained and the threshold of the DNA information molecule transmitter is studied. In-depth understanding of the characteristics of DNA information molecular transmitter is achieved which points out the specific direction of optimization of our DNA information molecular transmitter,including:
  
  $\bullet$The influence of the intensity and duration of the excitation voltage signal on the transmission process of DNA information molecules is discussed. Without considering the thermal effect of electrolysis, the higher the excitation voltage is, the stronger the ability of the system to resist the intersymbol interference will be. The shorter the excitation time is, the stronger the system's ability to resist the intersymbol interference will be. The excitation signal needs to have the characteristics of short duration and high excitation voltage, thus a high voltage pulse function is an ideal signal source.
  
  $\bullet$The influence of receiver position on system performance is discussed. The closer the receiver is to the cathode, the shorter the time for DNA concentration to reach the peak value, and the greater the peak intensity. The shorter the receiver distance, the better the signal transmission effect. Under the stimulation of short-term and long-term signals, the peak of DNA concentration will arrive earlier at the same position, and the peak will be more prominent than the whole curve. At a distance of 0.5mm from the cathode, we can reach the maximum communication rate of 0.909hz; at a distance of 4.5mm from the cathode, we can reach the maximum communication frequency of 0.2778hz, while at a distance of 9.5mm from the cathode, the maximum communication frequency is only 0.05348hz. The data shows that the communication system based on free diffusion is not suitable for long-distance communication.
  
  $\bullet$When the receiver is placed within the range of 5mm from the transmitter, it can achieve a faster transmission speed and strong anti-interference ability. After the range of 7mm, the signal interval must be increased to tens of seconds to suppress the inter code interference, and its ability to resist the inter code interference will also be reduced. 
  Considering the equilibrium constant, in the presence of OH-, the hydrolysis reaction of Zr4+ will occur. Because the reaction speed is proportional to the fourth power of COH -, the reaction speed is extremely slow at low OH - concentration. Under the observed time scale, it can be considered that the reaction will not occur without applying the excitation voltage. The voltage threshold of electrochemical reaction should be 2.17v. However, under the actual conditions, due to the influence of electrode impedance, solution conductivity and other factors, more voltage may be needed to start the reaction. By controlling the excitation voltage and duration, a DNA / Zr4 + monolayer can be released many times. Under the signal of excitation voltage of 5V and excitation duration of 1s, each layer of Zr4 + / DNA monolayer can be released six times at most.


\end{digest}

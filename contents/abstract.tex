% !TEX root = ../main.tex

\begin{abstract}
  SMC(Synthetic Molecular communications,合成分子通信)是一个充满前景的研究热点,
  它在微小尺度上的物质运输与信息传输上展现出了无限的潜力,适用于药物传输的精细控制
  、基因工程、人工合成蛋白质等领域。信号调制器是SMC系统的的重要组成部分,本文的研究基于
  一种微尺度SMC调制器。该调制器采用LbL(Layer-by-layer,逐层)自组装技术在金薄膜表面固定DNA,
  通过电化学反应实现LbL结构的分解与DNA的受控释放,将电信号转换为DNA分子信号。
  该SMC调制器在先前的实验中
  已成功的实现了将二进制信号向DNA分子信号的转换,通信速度可以达到1$bit/min$。

  目前还没有研究者对该LbL结构在受电信号控制下分解并释放DNA这一过程进行完整的数学建模。
  将电化学反应模型应用于对LbL结构的分解过程的研究对DNA受控释放的原理的理解有一定的指导意义,
  使用扩散模型对DNA释放后的运动进行仿真,可以描述DNA分子信号传输过程,对SMC系统的信道与解调器设计
  提供数据支持。
  并可以根据仿真结果,对SMC系统进行系统优化。

  本文利用COMSOL Multiphysics与MATLAB建立了该通信过程的三维模型,进行了
  电化学仿真、化学动力学仿真以及扩散过程仿真。主要内容如下:

  1)简要介绍了分子通信的基本原理与发展现状,包括利用LbL技术对DNA分子进行固定这一技术。
  介绍了课题组的先前实验,说明了本文的研究意义与研究内容。介绍了COMSOL Multiphysics这一软件,
  并列举了其在分子通信领域的运用实例。
  
  2)介绍了DNA/$\ce{Zr^4+}$LbL结构的组装过程以及理化性质。对该SMC系统的通信原理进行了探究,
  描述了通信过程中的电化学反应、水解反应、电迁移及扩散过程。并对描述反应过程与粒子运动的
  有关方程进行了详细说明。

  3)在COMSOL Multiphysics中对通信系统进行了三维建模,对上述三个过程进行了仿真,得到了
  与先前实验相符的DNA浓度曲线,验证了对通信过程进行建模仿真可行性。

  4)对通信系统的系统特征进行研究。讨论了激励信号的强度与持续时间对通信过程的影响讨,
  信道中不同位置的浓度曲线以及系统的阈值。为SMC系统设计中的参数选择提供了依据。


\end{abstract}

\begin{abstract*}
  Shanghai Jiao Tong University (SJTU) is a key university in China. SJTU was
  founded in 1896. It is one of the oldest universities in China. The University
  has nurtured large numbers of outstanding figures include JIANG Zemin, DING
  Guangen, QIAN Xuesen, Wu Wenjun, WANG An, etc.

  SJTU has beautiful campuses, Bao Zhaolong Library, Various laboratories. It
  has been actively involved in international academic exchange programs. It is
  the center of CERNet in east China region, through computer networks, SJTU has
  faster and closer connection with the world.
\end{abstract*}

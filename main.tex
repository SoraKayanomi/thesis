% !TeX encoding = UTF-8

% 载入 SJTUThesis 模版
\documentclass[degree=bachelor, zihao=-4]{sjtuthesis}
% 选项
%   degree=[doctor|master|bachelor|course],   % 可选(默认:doctor),学位类型
%   zihao=[-4|5],                             % 可选(默认:5),正文字号大小
%   language=[chinese|english],               % 可选(默认:chinese),论文的主要语言
%   review,                                   % 可选(默认:关闭),盲审模式
%   [twoside|oneside]                         % 可选(默认:twoside),单双页模式

% 论文基本配置,加载宏包等全局配置
% !TEX root = ./main.tex

\sjtusetup{
  %
  %******************************
  % 注意:
  %   1. 配置里面不要出现空行
  %   2. 不需要的配置信息可以删除
  %******************************
  %
  % 信息录入
  %
  info = {%
    %
    % 标题
    %   可使用“\\”命令手动控制换行
    %
    title           = {DNA分子通信系统的系统模型研究},
    title*          = {A Study of Systems Modeling in DNA-Based Molecular Communications},
    %
    % 关键词
    %
    keywords        = {DNA分子, COMSOL, 电场, Butler-Volmer},
    keywords*       = {DNA molecule, COMSOL, Electric field, Butler-Volmer},
    %
    % 姓名
    %
    author          = {孙\quad{}文韬},
    author*         = {Wentao\quad{}Sun},
    %
    % 指导教师
    %
    supervisor      = {闫浩},
    supervisor*     = {Hao Yan},
    %
    % 副指导教师
    %
    % assisupervisor  = {某某教授},
    % assisupervisor* = {Prof. Uom Uom},
    %
    % 学号
    %
    id              = {516030910265},
    %
    % 学位
    %   本科生不需要填写
    %
    degree          = {工学学士},
    degree*         = {Bachelor of Engineering},
    %
    % 专业
    %
    major           = {测控技术与仪器},
    major*          = {Measurement and Control Technology and Instrument Science},
    %
    % 所属院系
    %
    department      = {仪器科学与工程系},
    department*     = {Depart of Instrument Science and Engineering},
    %
    % 课程名称
    %   仅课程论文适用
    %
    coursename      = {某某课程},
    %
    % 答辩日期
    %   使用 ISO 格式;默认为当前时间
    %
    % date            = {2014-12-17},
    %
    % 资助基金
    %
    % fund  = {
    %           {国家 973 项目 (No. 2025CB000000)},
    %           {国家自然科学基金 (No. 81120250000)},
    %         },
    % fund* = {
    %           {National Basic Research Program of China (Grant No. 2025CB000000)},
    %           {National Natural Science Foundation of China (Grant No. 81120250000)},
    %         },
  },
  %
  % 格式设置
  %
  format = {%
    %
    % 本科论文页眉 logo 颜色
    %   默认为黑色
    %
    % header-logo-color = red,
  },
  %
  % 名称设置
  %
  name = {%
    % publications      = {攻读学位期间完成的论文},
  },
}

% 参考文献支持宏包
\usepackage[backend=biber,style=gb7714-2015,gbpub=false,gbpunctin=false]{biblatex}
% 导入参考文献数据库
\addbibresource{bibdata/thesis.bib}

% 定义图片文件目录与扩展名
\graphicspath{{figures/}}
\DeclareGraphicsExtensions{.pdf,.eps,.png,.jpg,.jpeg}

% 确定浮动对象的位置,可以使用 [H],强制将浮动对象放到这里(可能效果很差)
\usepackage{float}

% 固定宽度的表格
\usepackage{tabularx}

% 表格中支持跨行
\usepackage{multirow}

% 表格中数字按小数点对齐
\usepackage{dcolumn}
\newcolumntype{d}[1]{D{.}{.}{#1}}

% 附带脚注的表格
\usepackage{threeparttable}

% 算法环境宏包
\usepackage[ruled,vlined,linesnumbered]{algorithm2e}
% \usepackage{algorithm}

% 代码环境宏包
\usepackage{listings}
\lstnewenvironment{codeblock}[1][]
  {\lstset{style=lstStyleCode,#1}}{}

% 国际单位制宏包
\usepackage{siunitx}

% 定理环境宏包
\usepackage{ntheorem}
% \usepackage{amsthm}

% 绘图宏包
\usepackage{tikz}

% 化学方程式宏包
\usepackage[version=4]{mhchem}

%tikz宏包
\usepackage{tikz}

% 一些文档中用到的 logo
\usepackage{hologo}
\newcommand{\XeTeX}{\hologo{XeTeX}}
\newcommand{\BibLaTeX}{\textsc{Bib}\LaTeX}

% 借用 ltxdoc 里面的几个命令。
\def\cmd#1{\cs{\expandafter\cmd@to@cs\string#1}}
\def\cmd@to@cs#1#2{\char\number`#2\relax}
\DeclareRobustCommand\cs[1]{\texttt{\char`\\#1}}

\newcommand*{\meta}[1]{{%
  \ensuremath{\langle}\rmfamily\itshape#1\/\ensuremath{\rangle}}}
\providecommand\marg[1]{%
  {\ttfamily\char`\{}\meta{#1}{\ttfamily\char`\}}}
\providecommand\oarg[1]{%
  {\ttfamily[}\meta{#1}{\ttfamily]}}
\providecommand\parg[1]{%
  {\ttfamily(}\meta{#1}{\ttfamily)}}
\providecommand\pkg[1]{{\sffamily#1}}

% 自定义命令

% E-mail
\newcommand{\email}[1]{\href{mailto:#1}{\texttt{#1}}}

% hyperref 宏包在最后调用
\usepackage{hyperref}


\begin{document}

%TC:ignore

% 无编号内容:中英文论文封面、授权页
\maketitlepage
\makeorigpage*
\makeauthpage[scans/authorization.pdf]

% 使用罗马数字对前言编号
\frontmatter

% 摘要
% !TEX root = ../main.tex

\begin{abstract}
  SMC(Synthetic Molecular communications,合成分子通信)是一个充满前景的研究热点,
  它在微小尺度上的物质运输与信息传输上展现出了无限的潜力,适用于药物传输的精细控制
  、基因工程、人工合成蛋白质等领域。信号调制器是SMC系统的的重要组成部分,本文的研究基于
  一种微尺度SMC调制器。该调制器采用LbL(Layer-by-layer,逐层)自组装技术在金薄膜表面固定DNA,
  通过电化学反应实现LbL结构的分解与DNA的受控释放,将电信号转换为DNA分子信号。
  该SMC调制器在先前的实验中
  已成功的实现了将二进制信号向DNA分子信号的转换,通信速度可以达到1$bit/min$。

  目前还没有研究者对该LbL结构在受电信号控制下分解并释放DNA这一过程进行完整的数学建模。
  将电化学反应模型应用于对LbL结构的分解过程的研究对DNA受控释放的原理的理解有一定的指导意义,
  使用扩散模型对DNA释放后的运动进行仿真,可以描述DNA分子信号传输过程,对SMC系统的信道与解调器设计
  提供数据支持。
  并可以根据仿真结果,对SMC系统进行系统优化。

  本文利用COMSOL Multiphysics与MATLAB建立了该通信过程的三维模型,进行了
  电化学仿真、化学动力学仿真以及扩散过程仿真。主要内容如下:

  1)简要介绍了分子通信的基本原理与发展现状,包括利用LbL技术对DNA分子进行固定这一技术。
  介绍了课题组的先前实验,说明了本文的研究意义与研究内容。介绍了COMSOL Multiphysics这一软件,
  并列举了其在分子通信领域的运用实例。
  
  2)介绍了DNA/$\ce{Zr^4+}$LbL结构的组装过程以及理化性质。对该SMC系统的通信原理进行了探究,
  描述了通信过程中的电化学反应、水解反应、电迁移及扩散过程。并对描述反应过程与粒子运动的
  有关方程进行了详细说明。

  3)在COMSOL Multiphysics中对通信系统进行了三维建模,对上述三个过程进行了仿真,得到了
  与先前实验相符的DNA浓度曲线,验证了对通信过程进行建模仿真可行性。

  4)对通信系统的系统特征进行研究。讨论了激励信号的强度与持续时间对通信过程的影响讨,
  信道中不同位置的浓度曲线以及系统的阈值。为SMC系统设计中的参数选择提供了依据。


\end{abstract}

\begin{abstract*}
  Shanghai Jiao Tong University (SJTU) is a key university in China. SJTU was
  founded in 1896. It is one of the oldest universities in China. The University
  has nurtured large numbers of outstanding figures include JIANG Zemin, DING
  Guangen, QIAN Xuesen, Wu Wenjun, WANG An, etc.

  SJTU has beautiful campuses, Bao Zhaolong Library, Various laboratories. It
  has been actively involved in international academic exchange programs. It is
  the center of CERNet in east China region, through computer networks, SJTU has
  faster and closer connection with the world.
\end{abstract*}


% 目录、插图目录、表格目录
\tableofcontents
\listoffigures*
\listoftables*
\listofalgorithms*

% 主要符号、缩略词对照表
\input{contents/nomenclature}

%TC:endignore

% 使用阿拉伯数字对正文编号
\mainmatter

% 正文内容
\input{contents/literature_review}      %文献综述
% !TEX root = ../main.tex
\chapter{}
\section{}               %模型建立
\input{contents/simulation_1}           %电化学反应仿真(电解食盐水)
\input{contents/simulation_2}           %反应动力学反应仿真(Zr4+水解)
\input{contents/simulation_3}           %DNA电泳仿真
% !TEX root = ../main.tex

\begin{summary}
这里是全文总结内容。
近年来,分子通信研究高速发展,诞生了很多新理论与新技术。
一种基于DNA/$\ce{Zr^{4+}}$层层自组装结构在电信号控制下分解的技术
,实现了DNA的可控释放。本文使用COMSOL Multiphysics与MATLAB软件,
在课题组实验数据基础上,综合电化学反应原理、化学动力学原理、扩散原理
等对DNA的受控释放过程进行了建模与仿真,重点对激励信号与接收机位置
进行了研究,分析和总结了影响该系统通信速度、抗干扰能力、总发送量
这三个因素的影响。本文主要研究结果如下:

1)通过对分子通信的学习,了解了目前分子的发展现状:目前分子通信的系统领域已经有多种概念模型和基本框架被建立,
但在如何提高纳米机器的性能水平,如何更有效地传输信号分子,如何提升通信距离以及抗干扰能力这三个方面,
仍有很多挑战等待科研人员,

2)通过对有关化学理论的查阅与理解,本文提出了一种DNA/$\ce{Zr^{4+}}$
层层自组装结构的
分解的原理。在电信号的激励下,阴极处会发生式~\ref{equation:cathode_reaction}
所描述的化学反应,这使得阴极处$\ce{OH-}$浓度增加,促使$\ce{Zr^4+}$
发生式~\ref{Zr4+水解}描述的水解反应,$\ce{Zr^4+}$的消耗导致了
DNA/$\ce{Zr^{4+}}$层层自组装结构的分解,被释放到溶液中的DNA在
浓度梯度力与电场力的共同作用下,在溶液中扩散。

3)在COMSOL Multiphysics与MATLAB上对上述反应过程进行数值仿真,
在引用课题组实验的模型、参数的情况下结果较好的还原了实验数据。
验证了电极反应、水解反应、电泳过程共同作用的仿真可以适用于对
该分子通信系统的研究。

4)不考虑电解热效应的情况下,激励电压越高,系统的抗信道间干扰能力越强。
激励时长越短,系统的抗码间干扰能力越强。
激励信号应有持续时间较短、激励电压较高的特性,
高电压的脉冲函数是一个理想的信号源。

5)距离阴极越近的位置,DNA浓度达到峰值的时间越短,峰值强度越大。
系统通信距离越短,通信效果越好。
短时长的信号的激励下,相同位置处,DNA浓度峰值会更早到达,
峰值相对于整条曲线也会更突出。
在距离阴极0.5mm的地方,
我们可以达到0.909Hz的最大通信频率;在距离阴极4.5mm处,我们可以达到0.2778Hz的最大通信频率,
而在距离阴极9.5mm处,最大通信频率只有0.05348Hz。
数据说明了,基于自由扩散的通信系统,不适合远距离通信。接收机放置在距发送机5$mm$这一范围内时,
能达到一个较快的通信速度,而且抗干扰能力强。在超过7$mm$这一范围后,通讯的间隔必须增加到数十秒,
以抑制码间干扰,而且其抗信道间干扰能力也会下降。

6)综合考虑平衡常数,在$\ce{OH-}$存在时,$\ce{Zr^4+}$水解反应
都会发生,由于反应速度与$c_{\ce{OH-}}$的4次方成正比,所以低浓度时,反应速度极慢,在所观测的
时间尺度下,可以认为不施加激励电压时反应不发生。
电化学反应的电压阈值应为2.17$V$,但实际条件下,
由于电极阻抗、溶液电导率等因素的影响,可能需要更大的电压来启动此反应。
通过对激励电压与时长的控制,一个DNA/$\ce{Zr^4+}$单层可以被多次释放,
在激励电压为5$V$激励时长为1$s$的信号作用下每一层$\ce{Zr^4+}$/DNA单层最多可以释放6次。


\end{summary}
                %结论与展望

%TC:ignore

% 使用英文字母对附录编号
\appendix

% 附录内容,本科学位论文可以用翻译的文献替代。
\input{contents/app_maxwell_equations}
\input{contents/app_flow_chart}

% 文后无编号部分
\backmatter

% 参考资料
\printbibliography[heading=bibintoc]

% 用于盲审的论文需隐去致谢、发表论文、参与项目、申请专利、简历

% 致谢
% !TEX root = ../main.tex

\begin{acknowledgements}
  在论文写作的全程,我的导师闫浩对我进行了悉心的指导,通过线上会议及答疑的方式严格按照论文写作规划和指导要求认真负责的与我沟通,对我不胜其烦的问题给予积极耐心的回应。老师提出了诸多宝贵意见和建议,让我对毕业论文写作有了更深入全面的了解。在她的帮助和鼓励下,我的研究思路得以拓宽,论作水平得到了很大的提高。同时,老师的严谨治学的作风和认真负责的态度对我在论文写作中也产生了莫大的积极影响,在老师的带领下我以高度的重视和严谨的态度认真完成论文写作。对闫浩老师一直以来的帮助与鼓励,再次向她表达我由衷的感谢。
  
  同时,也要感谢四年以来上海交通大学的每一位任课老师,正是基于他们往日严谨负责的教学,使我学有所获,积累相关专业知识,为本次的论文写作打下扎实的基础。在此表达我诚挚的谢意和祝愿。
  
  然后,也要对陪我共度四年大学时光的朋友们给予诚挚感谢。Mogami Shizuka在精神上给了我莫大鼓励,在无助的时候给予我前行的动力。鸦教练、呆呆等雀士与我在Mahjong部共度了许多美好的时光。偶研社的
  社员们给我留下了诸多深刻的回忆。在此也十分感谢一直他们的相伴与鼓励。最后特别感谢我的女朋友在4年多来对我的无私陪伴,同我度过最美好的青春时光。
  
\end{acknowledgements}


% 发表论文、参与项目、申请专利、简历
% 盲审论文中,发表学术论文及参与科研情况等仅以第几作者注明即可,不要出现作者或他人姓名
%\input{contents/publications}
%\input{contents/achievements}
%\input{contents/resume}

% 中文学士学位论文要求在最后有一个英文大摘要,单独编页码,英文学士学位论文不需要
% !TEX root = ../main.tex

\begin{digest}
  to do
\end{digest}


%TC:endignore

\end{document}

% !TeX encoding = UTF-8

% 载入 SJTUThesis 模版
\documentclass[degree=bachelor, zihao=5]{sjtuthesis}
% 选项
%   degree=[doctor|master|bachelor|course],   % 可选(默认:doctor),学位类型
%   zihao=[-4|5],                             % 可选(默认:5),正文字号大小
%   language=[chinese|english],               % 可选(默认:chinese),论文的主要语言
%   review,                                   % 可选(默认:关闭),盲审模式
%   [twoside|oneside]                         % 可选(默认:twoside),单双页模式

% 论文基本配置,加载宏包等全局配置
% !TEX root = ./main.tex

\sjtusetup{
  %
  %******************************
  % 注意:
  %   1. 配置里面不要出现空行
  %   2. 不需要的配置信息可以删除
  %******************************
  %
  % 信息录入
  %
  info = {%
    %
    % 标题
    %   可使用“\\”命令手动控制换行
    %
    title           = {DNA分子通信系统的系统模型研究},
    title*          = {A Study of Systems Modeling in DNA-Based Molecular Communications},
    %
    % 关键词
    %
    keywords        = {DNA分子, COMSOL, 电场, Butler-Volmer},
    keywords*       = {DNA molecule, COMSOL, Electric field, Butler-Volmer},
    %
    % 姓名
    %
    author          = {孙\quad{}文韬},
    author*         = {Wentao\quad{}Sun},
    %
    % 指导教师
    %
    supervisor      = {闫浩},
    supervisor*     = {Hao Yan},
    %
    % 副指导教师
    %
    % assisupervisor  = {某某教授},
    % assisupervisor* = {Prof. Uom Uom},
    %
    % 学号
    %
    id              = {516030910265},
    %
    % 学位
    %   本科生不需要填写
    %
    degree          = {工学学士},
    degree*         = {Bachelor of Engineering},
    %
    % 专业
    %
    major           = {测控技术与仪器},
    major*          = {Measurement and Control Technology and Instrument Science},
    %
    % 所属院系
    %
    department      = {仪器科学与工程系},
    department*     = {Depart of Instrument Science and Engineering},
    %
    % 课程名称
    %   仅课程论文适用
    %
    coursename      = {某某课程},
    %
    % 答辩日期
    %   使用 ISO 格式;默认为当前时间
    %
    % date            = {2014-12-17},
    %
    % 资助基金
    %
    % fund  = {
    %           {国家 973 项目 (No. 2025CB000000)},
    %           {国家自然科学基金 (No. 81120250000)},
    %         },
    % fund* = {
    %           {National Basic Research Program of China (Grant No. 2025CB000000)},
    %           {National Natural Science Foundation of China (Grant No. 81120250000)},
    %         },
  },
  %
  % 格式设置
  %
  format = {%
    %
    % 本科论文页眉 logo 颜色
    %   默认为黑色
    %
    % header-logo-color = red,
  },
  %
  % 名称设置
  %
  name = {%
    % publications      = {攻读学位期间完成的论文},
  },
}

% 参考文献支持宏包
\usepackage[backend=biber,style=gb7714-2015,gbpub=false,gbpunctin=false]{biblatex}
% 导入参考文献数据库
\addbibresource{bibdata/thesis.bib}

% 定义图片文件目录与扩展名
\graphicspath{{figures/}}
\DeclareGraphicsExtensions{.pdf,.eps,.png,.jpg,.jpeg}

% 确定浮动对象的位置,可以使用 [H],强制将浮动对象放到这里(可能效果很差)
\usepackage{float}

% 固定宽度的表格
\usepackage{tabularx}

% 表格中支持跨行
\usepackage{multirow}

% 表格中数字按小数点对齐
\usepackage{dcolumn}
\newcolumntype{d}[1]{D{.}{.}{#1}}

% 附带脚注的表格
\usepackage{threeparttable}

% 算法环境宏包
\usepackage[ruled,vlined,linesnumbered]{algorithm2e}
% \usepackage{algorithm}

% 代码环境宏包
\usepackage{listings}
\lstnewenvironment{codeblock}[1][]
  {\lstset{style=lstStyleCode,#1}}{}

% 国际单位制宏包
\usepackage{siunitx}

% 定理环境宏包
\usepackage{ntheorem}
% \usepackage{amsthm}

% 绘图宏包
\usepackage{tikz}

% 化学方程式宏包
\usepackage[version=4]{mhchem}

%tikz宏包
\usepackage{tikz}

% 一些文档中用到的 logo
\usepackage{hologo}
\newcommand{\XeTeX}{\hologo{XeTeX}}
\newcommand{\BibLaTeX}{\textsc{Bib}\LaTeX}

% 借用 ltxdoc 里面的几个命令。
\def\cmd#1{\cs{\expandafter\cmd@to@cs\string#1}}
\def\cmd@to@cs#1#2{\char\number`#2\relax}
\DeclareRobustCommand\cs[1]{\texttt{\char`\\#1}}

\newcommand*{\meta}[1]{{%
  \ensuremath{\langle}\rmfamily\itshape#1\/\ensuremath{\rangle}}}
\providecommand\marg[1]{%
  {\ttfamily\char`\{}\meta{#1}{\ttfamily\char`\}}}
\providecommand\oarg[1]{%
  {\ttfamily[}\meta{#1}{\ttfamily]}}
\providecommand\parg[1]{%
  {\ttfamily(}\meta{#1}{\ttfamily)}}
\providecommand\pkg[1]{{\sffamily#1}}

% 自定义命令

% E-mail
\newcommand{\email}[1]{\href{mailto:#1}{\texttt{#1}}}

% hyperref 宏包在最后调用
\usepackage{hyperref}


\begin{document}

%TC:ignore

% 无编号内容:中英文论文封面、授权页
\maketitlepage
\makeorigpage*
%\makeauthpage[scans/authorization.pdf]

% 使用罗马数字对前言编号
\frontmatter

% 摘要
% !TEX root = ../main.tex

\begin{abstract}
  SMC(Synthetic Molecular communications,合成分子通信)是一个充满前景的研究热点,
  它在微小尺度上的物质运输与信息传输上展现出了无限的潜力,适用于药物传输的精细控制
  、基因工程、人工合成蛋白质等领域。信号调制器是SMC系统的的重要组成部分,本文的研究基于
  一种微尺度SMC调制器。该调制器采用LbL(Layer-by-layer,逐层)自组装技术在金薄膜表面固定DNA,
  通过电化学反应实现LbL结构的分解与DNA的受控释放,将电信号转换为DNA分子信号。
  该SMC调制器在先前的实验中
  已成功的实现了将二进制信号向DNA分子信号的转换,通信速度可以达到1$bit/min$。

  目前还没有研究者对该LbL结构在受电信号控制下分解并释放DNA这一过程进行完整的数学建模。
  将电化学反应模型应用于对LbL结构的分解过程的研究对DNA受控释放的原理的理解有一定的指导意义,
  使用扩散模型对DNA释放后的运动进行仿真,可以描述DNA分子信号传输过程,对SMC系统的信道与解调器设计
  提供数据支持。
  并可以根据仿真结果,对SMC系统进行系统优化。

  本文利用COMSOL Multiphysics与MATLAB建立了该通信过程的三维模型,进行了
  电化学仿真、化学动力学仿真以及扩散过程仿真。主要内容如下:

  1)简要介绍了分子通信的基本原理与发展现状,包括利用LbL技术对DNA分子进行固定这一技术。
  介绍了课题组的先前实验,说明了本文的研究意义与研究内容。介绍了COMSOL Multiphysics这一软件,
  并列举了其在分子通信领域的运用实例。
  
  2)介绍了DNA/$\ce{Zr^4+}$LbL结构的组装过程以及理化性质。对该SMC系统的通信原理进行了探究,
  描述了通信过程中的电化学反应、水解反应、电迁移及扩散过程。并对描述反应过程与粒子运动的
  有关方程进行了详细说明。

  3)在COMSOL Multiphysics中对通信系统进行了三维建模,对上述三个过程进行了仿真,得到了
  与先前实验相符的DNA浓度曲线,验证了对通信过程进行建模仿真可行性。

  4)对通信系统的系统特征进行研究。讨论了激励信号的强度与持续时间对通信过程的影响讨,
  信道中不同位置的浓度曲线以及系统的阈值。为SMC系统设计中的参数选择提供了依据。


\end{abstract}

\begin{abstract*}
  Shanghai Jiao Tong University (SJTU) is a key university in China. SJTU was
  founded in 1896. It is one of the oldest universities in China. The University
  has nurtured large numbers of outstanding figures include JIANG Zemin, DING
  Guangen, QIAN Xuesen, Wu Wenjun, WANG An, etc.

  SJTU has beautiful campuses, Bao Zhaolong Library, Various laboratories. It
  has been actively involved in international academic exchange programs. It is
  the center of CERNet in east China region, through computer networks, SJTU has
  faster and closer connection with the world.
\end{abstract*}


% 目录、插图目录、表格目录
\tableofcontents
%\listoffigures*
%\listoftables*
%\listofalgorithms* %算法

% 主要符号、缩略词对照表
%\input{contents/nomenclature}

%TC:endignore

% 使用阿拉伯数字对正文编号
\mainmatter

% 正文内容
% !TEX root = ../main.tex

%先介绍做课题的目的和背景,再引出需要用到COMSOL来进行仿真,再介绍COMSOL


\chapter{绪论}
\section{课题研究背景}
本节简要介绍了分子通信的基本原理与发展现状。
介绍了课题组的微观分子通信实验平台的基本原理,
阐明了了本毕业设计工作的研究内容与研究意义。
\subsection{纳米网络的简述}
随着纳米技术的高速发展,研究者们现在可以
在纳米至微米尺度范围内制造能完成特定功能的结构,
这种微小结构被称为纳米机器(nanomachine)。
由于尺寸较小,单个纳米机器的功能非常有限,只能执行简单的任务
\cite{Weiss_©2003},严重限制了其应用于发展。为了解决这个问题,
纳米机器网络(简称纳米网络,nanonetwork)的概念被提出,
它将多个纳米机器互联、传递信息和彼此协调,扩展了纳米机器在信息采集、运算、储存
等方面的能力,进而使得整个纳米机器群体在更广的覆盖范围内执行更复杂和更精准的任务。
纳米网络在医疗卫生、工业制造、环境治理、国防建设等诸多领域有良好发展前景,
将广泛服务于我们的生产和生活中。

分子通信能够在复杂的生物环境和生物体中实现稳定而且可靠的信息传递,
同时具有生物兼容性好、不受收发器体积和能耗等限制等优点,被认为是组建
纳米网络,尤其是组建面向生物应用的纳米网络,最可行的通信方案之一
\cite{Suda05exploratoryresearch}。
\subsection{分子通信的基本原理}
分子通信(molecular communication)
指的是以生物化学分子作为信息载体的通信技术\cite{Hiyama2010Molecular}。

分子通信作为地球上最古老、最普遍的通信机制之一,广泛存在于自然界。
无论是对简单的单细胞生物,还是对复杂的多细胞动植物来说,分子通信都是维持它们生命必不可少的一环。
例如,许多细菌都会对它们邻居分泌的信号分子(information molecular)做出反应,以协调彼此的行为,并影响它们自身的运动、
产生抗生素、生成孢子等行为,这被成为群体感应。同样的,信号分子(例如信息素)
也广泛的存在于从低等的昆虫到高等的哺乳动物的日常交流中,并深深的影响了它们的行为。
信息素由个体释放,并指导群体里的其他个体前往觅食地,警告同伴有潜在的危险,
以及协调其他各种行为。此外,在多细胞动物体内,细胞与细胞之间也通过信号分子进行通信,
以完成相应的生理功能。例如,我们人体的神经系统中的电信号的传递,就是由神经递质(一种信号分子)
的释放与接收来完成的;在内分泌系统中,内分泌系统会向循环系统释放激素分子,
它作为一种信号分子被远端的目标细胞(靶细胞)所接收,从而完成细胞间的通信,调控靶细胞的行为\cite{Atakan2014Molecular}。

人工分子通信系统基于这一种在自然界就广泛存在的通信方式,
对传统的通信设备进行改造,以生物化学分子作为信息载体
在发射器和接收器之间进行通信。
如图
~\ref{fig:molecular_communication_example}
所示,一个典型的分子通信的典型过程包含三个过程
\cite{基于扩散的分子通信与身体域纳米网络}:
1)发送器(Transmitter)
生成携带着特定编码信息的信息分子(Information Molecule);
2)被释放的信息分子通过流体(液体或气体)介质传送到接又称接收器(Receiver);
3)接收机基于接受到的信息分子的物理或化学特性,对信息进行解码。
\begin{figure}[H]
    \centering
    \includegraphics[scale=0.8]{simple model of molecular communication.png}
    \caption{分子通信示意图\cite{compic}}
    \label{fig:molecular_communication_example}
\end{figure}

为方便叙述,本文中“分子通信”特指人工分子通信,而非天然分子通信。

\subsection{分子通信的发展现状与面临挑战}
分子通信这个概念自2005年被提出以来\cite{Suda05exploratoryresearch},
就受到了相关领域研究者的密切关注,发展迅速,
在信道模型与容量分析、调制与解调技术、
信号检测技术、构架和协议设计等方面
已经有多种概念模型和基本框架被建立,
目前已经成为了纳米网络通信机制中非常重要的研究方向。

目前已经有研究成功的
实现了宏观的分子通信实验\cite{10.1007/978-81-322-1007-8_56},这证明了宏观分子通信的可行性。
然而建立纳米网络需要基于微观的分子通信系统,微观分子通信不单单是宏观分子通信
的缩小化,它们之间还存在较大的差别,因此宏观的分子通信实验还不能解决
纳米网络对微观分子通信的实验需求。

总的来说,分子通信的研究主要都集中在理论方面,缺乏实验研究,技术成熟度低。
造成这种现状的最主要的瓶颈问题是
缺少能执行稳定而连续信息传递的微观分子通信平台,使得大量的理论研究
成功无法通过实验进行验证,更进一步阻碍了纳米网络从理论向实际应用的推进,严重
限制了该领域的发展。所以,尽快突破微观分子通信实验平台这个瓶颈问题,已成为纳米网络与分子通信领域
的关注核心和基本共识\cite{Liu1999}。

其中DNA分子通信指的是以DNA为信息分子的分子通信系统。
在当前纳米机器研究中,DNA纳米机器是其中最具有可能迈向第三代
包含人工智能和纳米计算机在内的纳米机器种类\cite{McCutcheon2017}。
因此本课题组正在开展以DNA为信息载体、
以DNA纳米机器为通信主体的微观分子通信实验平台的研究,
为纳米网络提供第一个完整的微观分子通信实验平台,解决本领域瓶颈问题。

\section{课题研究目的及意义}
本毕业设计立足于课题组的现有的DNA分子通信实验平台,对其发送器部分开展理论研究。
我们实现的DNA信息分子的发送器,是一个宏观电信号到微观DNA信号的转化接口。
其主要原理是将锆离子$\ce{Zr^4+}$作为粘合剂胶水,
采用LbL(Layer-by-Layer,逐层)自组装技术在金薄膜表面固定完成多层DNA/$\ce{Zr^4+}$结构。
通过电化学反应执行多层DNA结构的分解与可控释放。
本课题组在先前实验中已经构建完成了基于DNA/$\ce{Zr^{4+}}$LbL自组装结构的
分子通信系统,该系统在外部电压控制下可以将DNA分子从发送机上释放到溶液中。
但对于该分子通信系统的系统模型研究还未开展。

而在所有理论研究中,系统模型是其他理论研究的基础。
首先,虽然目前大量不同的分子通信的理论模型被提出。
但是当前系统模型的研究成果均未经过实验验证其有效性。
更为重要的是,由于DNA的释放与接收机制与以往系统的不同,
因此本项目的DNA微观分子通信的系统模型不同于之前其他
微观分子通信的系统模型。
又由于领域内尚未开展DNA分子通信的研究,
因此需要根据DNA信息分子的释放与接收机制特点,
开展DNA微观分子通信系统模型的研究。

所以本文立足于实验平台,结合DNA微观分子通信特征,开展系统建模的研究。
在课题组实验的基础上,探究DNA受控释放的原理。
综合分子通信理论、电化学反应原理、化学动力学原理、扩散原理
等对DNA的受控释放过程进行建模与仿真。
根据仿真结果对影响该分子通信系统性能的参数进行
研究,并调整参数进行系统的优化。

该研究可以解决目前大量纳米网络与分子通信的理论研究亟需通过实验验证的问题,促
进纳米网络与分子通信的理论研究沿着正确的方向发展。
其次,也将为纳米网络与分子通信从理论研究向实际应用的迈进奠定关键的一步。
因此,无论从理论上还是实践上,本项目的研究都具有重要意义。           %引言
% !TEX root = ../main.tex
\chapter{系统的通信原理}

\section{DNA/$\ce{Zr^{4+}}$LbL结构}
\subsection{组装过程}
多层膜的制造过程包括
三步:清洗金薄膜,孵化金
薄膜和$\ce{Zr^{4+}}$/ssDNA的循环沉积
直到达到所需的层数为止,如图~\ref{fig:incubation_process}所示。

1)清洗金薄膜:使用去离子水对金薄膜进行除尘,
然后将金薄膜放入丙酮($\ce{C3H6O}$)中煮沸10分钟,
再次用去离子水清洗。再将金薄膜放入无水乙醇中煮沸10分钟,
再次用去离子水清洗。

2)孵化金薄膜:将清洗后的金薄膜MUA(巯基十一酸)溶液溶液中浸泡一晚后,
再次使用去离子水清洗。

3)$\ce{Zr^{4+}}$/ssDNA的循环沉积:将金薄膜浸入$0.1mM$的$\ce{Zr^{4+}}$溶液5分钟,
再浸入$0.1mg/mL$的ssDNA溶液5分钟,接着使用纯化水冲洗,即可得到$\ce{Zr^{4+}}$/ssDNA
双层结构,重复这一过程至所需的层数。最后,使用氮气对组装好的多层膜进行干燥。

\begin{figure}[H]
    \centering
    \includegraphics[scale=0.5]{Incubation process and repeated cyclic deposition process.jpg}\\
    1)MUA乙醇溶液 2)纯化水 3)$\ce{Zr^{4+}}$溶液 \\
    4)ssDNA溶液 5)氮气\\
    \caption{DNA/$\ce{Zr^{4+}}$LbL结构组装过程}
    \label{fig:incubation_process}
\end{figure}

\subsection{理化性质}
根据扫描隧道显微镜(STM)的扫描结果,组装成的多次膜结构中,$\ce{Zr^{4+}}$在
平面上均匀分布,根据信号峰值的间距可以得到$\ce{Zr^{4+}}$之间的距离为3~4$nm$
,$\ce{Zr^{4+}}$/ssDNA的厚度约为5$nm$\cite{Shervedani2011Electrochemical}。

电化学特性方面,根据实验测定\cite{Karimi2011}多层膜结构组成的电极的电荷转移电阻为$136.5\pm1.8k\Omega$,
双电层电容为$1.89\pm0.06\mu F$,赝电容为$0.46\pm0.04\mu F$。


\section{通信过程}
\subsection{电信号激励下的电化学反应}
标准电极电位的大小反映了电极在进行电极反应时,相对于标准氢电极的得失电子的能力。
标准电极电位的值越大,就越容易获得电子,换句话说它的氧化性就越强;反之,
标准电极电位的值越小,就越容易失去电子,换句话说它的还原就越强\cite{Trasatti1980}。
将常见的物质的标准电极电位排列得到一个序列,称为电化学序(Electrochemical Series)。

由于溶液环境中,主要成分是氯化钠($\ce{NaCl}$)。根据溶液里潜在反应物质的电化学序提供的信息\cite{Weik2001}
\footnote{在实验用溶液中初始$\ce{pH=7}$,即在溶液中$c_{\ce{OH-}}=c_{\ce{H+}}=10^{-7}mM$
这时由于$\ce{OH-}$与$\ce{H+}$的浓度处于一个极低的水平,我们认为是$\ce{H2O}$参与了反应,
所以采用了$\ce{H2O}$的标准电极电位。}:
\begin{align}
\label{equation:anode_reaction} \ce{Cl2(g) + 2e- &<=> 2 Cl-} \quad\quad\quad\quad &{E°}=1.35827V \\
\ce{O2(g) + 4H+ + 4 e- &<=> 2H2O} &{E°}=1.229V \\ 
\label{equation:cathode_reaction} \ce{2 H2O + 2 e- &<=> H2(g) + 2OH- } &{E°}=-0.8277V \\
\ce{Zr^4+ + 4e- &<=> Zr}   &{E°}=-1.45V \\
\ce{Na+ + e- &<=> Na }     &{E°}=-2.71V
\end{align}


在施加电信号的时,阴极处,$\ce{H2O}$分子的氧化性强于$\ce{Zr^4+}$离子与$\ce{Na+}$离子
($E°=–0.8277V>-1.45V>–2.71V$),所以释放的电子会优先被$\ce{H2O}$分子获取,反应生成
$\ce{H2}$气体与$\ce{OH-}$离子;阳极处,$\ce{Cl-}$离子的还原性强于$\ce{H2O}$分子
($E°=-1.35827V<-1.229V$),所以$\ce{Cl-}$离子会优先释放电子,反应生成$\ce{Cl2}$气体。

\subsection{Zr$^{4+}$水解过程}
考虑如下的反应式:
\begin{equation}
    \ce{\alpha{A} +\beta{B} <=> \rho{R} + \sigma{S}}
\end{equation}

对某一可逆反应,在一定温度下,反应达到平衡状态后,反应物与生成物热力学活性的系数次方的比是一个常数,
称为化学反应的平衡常数$K°$(equilibrium constant):
\begin{equation}
    K°=\frac{\{R\}^\rho \{S\}^\sigma}{\{A\}^\alpha \{B\}^\beta}
\end{equation}

其中,{$\{X\}$}表示的是物质$X$在平衡状态下的热力学活性,基于底物的浓度。
研究者们发现实际反应与理论值存在有偏差的情况,这时我们使用活性系数修正底物的浓度,
以弥补偏离理想的情况\cite{Königsberger2017}。
在本文所涉及的反应中,均可以将热力学活性数值上等同于浓度。平衡常数越大,则正向反应越彻底。

对于水,会有如下可逆反应:
\begin{equation}
    \ce{H2O <=> H+ + OH-} \quad K°=K_w
\end{equation}

我们称之为水的电离反应,其中$K_w$被称为水的离子积,随温度变化而变化,在$25°C$时,$k_w=10^{-14}$。
它表明了在水溶液中,温度一定的情况下,溶液环境里的$c_{\ce{OH-}}$与$c_{\ce{H+}}$的乘积是一个定值。

如反应式 ~\ref{equation:cathode_reaction} 所示,随着电化学反应的进行,在阴极区域会不断产生$\ce{OH-}$
离子,从而抑制水的电离,降低了$c_{\ce{H+}}$。

考虑$\ce{Zr^4+}$的水解反应\cite{Thoenen2004Development}:
\begin{equation}
    \ce{Zr^4+ +2 H2O  = ZrO2 + 4H+} \quad lgK°=1.9
    \label{Zr4+水解}
\end{equation}

$c_{\ce{H+}}$的降低会促进水解反应的发生,这会导致固定在LbL结构中的$\ce{Zr^4+}$不断水解成$\ce{ZrO2}$,
从而引起LbL结构的分解,并释放出DNA分子。

\subsection{电场影响下的DNA分子运动}
DNA分子的骨架由大量的磷酸基团脱水缩合连接,磷酸骨架的每个残基带有一个负电荷,所以DNA分子整体是带负电荷的
\cite{Lipfert2014}。在电信号的激励下,DNA会向远离极板的方向移动;而$\ce{Zr^4+}$带正电荷,会向靠近极板的方向
移动,并阻碍DNA分子的扩散。所以只有在$\ce{Zr^4+}$被水解反应消耗后,DNA才会被释放。

\subsection{通信过程总述}
在电信号的激励下,阴极处会发生式~\ref{equation:cathode_reaction}
所描述的化学反应,这使得阴极处$\ce{OH-}$浓度增加,促使$\ce{Zr^4+}$
发生式~\ref{Zr4+水解}描述的水解反应,$\ce{Zr^4+}$的消耗导致了
DNA/$\ce{Zr^{4+}}$LbL结构的分解,被释放到溶液中的DNA在
浓度梯度力与电场力的共同作用下,在溶液中扩散。这会引起反应容器中
固定位置处的DNA浓度随时间而变化。
%缺图

\section{基本控制方程}
\subsection{扩散定律}
菲克定律由阿道夫·菲克于十九世纪提出,对扩散给出了最简单的描述:
\begin{itemize}[leftmargin=40pt]
    \item [1)]由扩散引起的摩尔通量与浓度梯度成正比。
    \item [2)]空间上某一点的浓度变化率与浓度的空间二阶导数成正比。
\end{itemize}

菲克第一定律可以表示为:
\begin{equation}
    \mathbf{N_i}=-D_i\nabla{c_i}
\end{equation}

对于物质$i$,$\mathbf{N_i}$是摩尔通量($mol\cdot{m^{-2}\cdot{s^{-1}}}$),$D_i$是扩散系数($m^2\cdot{s^{-1}}$),$c_i$是浓度($mol\cdot{m^{-3}}$)。
根据连续性方程:
\begin{equation}
    \frac{\partial c_i}{\partial t}+\nabla\cdot{N_i}=0
    \label{eqation:continuity}
\end{equation}

可以推导出菲克第二定律:
\begin{equation}
    \frac{\partial c_i}{\partial t}=D_i\nabla^2{c_i}
    \label{fick's 2nd}
\end{equation}

在稀溶液中,我们假设$D_i$是一个常数。它描述了自由扩散时,浓度随时间的变化关系\cite{Sakaguchi2018} 。
\subsection{Nernst-Einstein 关系}
考虑外力作用于扩散粒子:
\begin{equation}
    \mathbf{v_d}=m_{abs}\mathbf{F}
\end{equation}

在这一本构关系中,$\mathbf{v_d}$表示漂移速度($m\cdot{s^{-1}}$),
$m_{abs}$是绝对迁移率($N\cdot{s\cdot{m^{-1}}}$),$F$是作用力($N$)。
考虑浓度梯度力同时作用于粒子,粒子的总通量$\mathbf{N_i}$可以表达为:
\begin{equation}
    \mathbf{N_i}=-D_i\nabla{c_i}+m_{i,abs}c_i\mathbf{F}
\end{equation}

当总通量为$0$时,扩散作用于迁移作用引起的通量大小相等且方向相反:
\begin{equation}
    \nabla{c_i}=\frac{m_{i,abs}}{D_i}c_i\mathbf{F}
\end{equation}

各个方向上的净通量为$0$时,可以假设扩散与迁移达到平衡。此时的浓度分部可以用Boltzmann方程来描述:
\begin{equation}
    c_i=c_{i,0}exp(-\frac{U}{kT})
\end{equation}

其中,$c_{i,0}$表示势能为零时的浓度($mol\cdot{m^{-3}}$),$U$表示分子的势能($J$),
$k$表示Boltzmann常数($J\cdot{K^{-1}}$),T是绝对温度(K)。此时该浓度场的梯度为:
\begin{equation}
    \nabla{c_i}=-c_{i,0}exp(-\frac{U}{kT})\frac{1}{kT}\nabla{U}
\end{equation}

根据定义,力是势能的负梯度:
\begin{equation}
    \mathbf{F}=-\nabla{U}
\end{equation}

结合等式(2-7)与(2-9):
\begin{equation}
    \frac{m_{i,abs}}{D_i}c_i\mathbf{F}=\frac{1}{kT}c_i\mathbf{F}
\end{equation}

化简后得到:
\begin{equation}
    m_{i,abs}=\frac{D_i}{kT}
    \label{Nernst-Einstein}
\end{equation}

该等式描述了带点粒子的扩散系数$D_i$与扩散迁移率$m_{i,abs}$的关系,被称为Nernst-Einstein关系\cite{Mehrer2007,CONWAY1972250}。
\subsection{Nernst-Planck 方程}
在电场作用下,带电粒子受到电场力$\mathbf{F}$:
\begin{equation}
    \mathbf{F}=-z_ie_0\nabla\phi
\end{equation}

其中,$z_i$表示粒子的电荷数,$e_0$是电子的基本电荷,$\phi$是电势($V$),$−\nabla{\phi}$ 表示电场。
我们定义$u_i=e_0m_{abs}$为该粒子的电化学迁移率,此时,漂移速度可由下式求得:
\begin{equation}
   \mathbf{v_d}=-z_iu_i\nabla\phi
\end{equation}

由此得到的离子$i$的迁移通量是漂移速度$v_d$和离子浓度$c_i$的乘积,此通量的贡献称为离子迁移或电迁移:
\begin{equation}
    \mathbf{N_{i,migr}}=-z_iu_ic_i\nabla\phi
\end{equation}

在一般的稀释电解质中,通量贡献可能有以下三种来源:扩散、迁移和对流:
\begin{equation}
    N_i=-\overbrace{D_i\nabla{c_i}}^{Diff}-\overbrace{z_iu_ic_i\nabla\phi}^{Migr}+\overbrace{c_i\mathbf{u}}^{Conv}
\end{equation}

其中$\mathbf{u}$是电解质速度($m\cdot{s^{-1}}$)。由Nernst-Einstein关系我们得到电化学迁移率$u_i$与扩散率$D_i$的关系。
代入到式(2-2)的连续性方程:
\begin{equation}
    \frac{\partial c_i}{\partial t}=\nabla{(D_i\nabla{c_i}+\frac{D_iz_ie_0}{kT}c_i\nabla\phi+c_i\mathbf{u})}
    \label{equation:Nernst_Planck}
\end{equation}

该等式描述了流体介质中带电化学物质的运动情况,被称为Nernst–Planck方程\cite{Mehrer2007}。
\subsection{离子迁移}
电解质中的电流密度$\mathbf{i}$可以通过该电解质中所有离子的贡献之和求得:
\begin{equation}
    \mathbf{i}=F\sum_i{z_i\mathbf{N_i}}
\end{equation}

式中,$F$为法拉第常数。将式(2-15)的通量代入到该方程,可以得到:
\begin{equation}
    \mathbf{i}=F\sum_i(-z_iD_i\nabla{c_i}-z_i^2u_ic_i\nabla\phi)+F\sum_iz_ic_i
\end{equation}

在大多数电化学反应中,除双电层区域外,都可以假设电解质呈电中性:
\begin{equation}
    F\sum_iz_ic_i=0
    \label{equation:neutrality}
\end{equation}

对流使得浓度保持均匀的分布,因此,在靠近电极的区域外,电解槽中的任何位置都具有恒定的电导率$\kappa$,电流密度表达式变为:
\begin{equation}
    \mathbf{i}=-\overbrace{F\sum_i(z_i^2u_ic_i)}^{Conductivity,\kappa}\nabla\phi
\end{equation}

这个方程表明,组成恒定的电解质中的电流完全由迁移产生。电流遵循欧姆定律,电导率由电解质中每个组成离子的迁移贡献总和决定\cite{Smedley1980}。

\subsection{化学反应速率方程}
考虑一个典型的化学反应:
\begin{equation}
    \ce{aA + bB -> pP + qQ}
\end{equation}

式中a,b,c,d为化学计量系数,A,B表示反应物,C,D表示产物。
根据IUPAC's Glod Book的定义\cite{GlossaryoftermsusedinphysicalorganicchemistryIUPACRecommendations1994},
在容积不变的封闭系统中发生的化学反应,在不考虑中间产物积累的情况下其速率$v$的表达式为:
\begin{equation}
    v=-\frac{1}{a}\frac{\mathrm{d}A}{\mathrm{d}t}
    =-\frac{1}{b}\frac{\mathrm{d}B}{\mathrm{d}t}
    =\frac{1}{p}\frac{\mathrm{d}P}{\mathrm{d}t}
    =\frac{1}{q}\frac{\mathrm{d}Q}{\mathrm{d}t}
\end{equation}

其中,$[X]$表示物质$X (X=A,B,P,Q)$的浓度($mol\cdot{m^{-3}}$)
速率方程是化学动力学中使用的数学表达式,用于将反应速率与每种反应物的浓度联系起来。对于体积恒定的封闭系统,通常采用以下形式:
\begin{equation}
    v=k[A]^n[B]^m
    \label{rate_equation}
\end{equation}

其中,$k$表示反应速率常数,与温度、离子活度、光照、固体反应物的接触面积、反应活化能等因素有关,通常可通过阿累尼乌斯方程计算出来,也可通过实验测定。
指数$n,m$为反应级数,取决于反应历程。在基元反应\footnote{基元反应是指在反应中一步直接转化为产物的反应,又称为简单反应。}中,
反应级数等于化学计量数。但在非基元反应中,反应级数与化学计量数不一定相等\cite{Witelski2015}。
\subsection{Arrhenius 方程}
Arrhenius 方程描述了温度如何影响化学反应速率的:
\begin{equation}
    k=Ae^{\frac{-E_a}{RT}}
    \label{equation:Arrhenius}
\end{equation}

其中,$k$是化学反应速率常数;$T$是绝对温度;$A$是指数前因子,是每个化学反应的固有常数;$E_a$是活化能;R是气体常数\cite{ref1}。


\subsection{Butler-Volmer 方程}
考虑一个简单的电化学反应:
\begin{equation}
    \ce{O + ne- -> R}
\end{equation}

根据法拉第电解定律,正向反应速率$v_f$与逆向反应速率$v_b$与电解质电流密度之间有以下关系:
\begin{equation}
    \begin{aligned}
        v_f=k_fc_o=j_f/nF\\
        v_b=k_bc_r=j_b/nF          %&是用于标注需要对齐的位置
    \end{aligned}
\end{equation}

其中,$c_o$与$c_r$分别是氧化分子与还原分子的浓度。由此推出净反应速度$v$与净电流密度$j$的关系:
\begin{equation}
    v=v_b-v_f=\frac{j_b-j_f}{nF}=\frac{j}{nF}
\end{equation}

Butler-Volmer 方程考虑了电极电位对反应物吉布斯自由能(Gibbs energy)的影响。
图~\ref{fig:ButlerVolmerGibbsPlot}绘制了不同物质吉布斯能量曲线。
反应坐标$\xi$是一种距离的量度,电极在左边,本体溶液在右边。
蓝色能量曲线显示,当一个氧化分子靠近电极表面时,当没有施加电位时,
它的吉布斯能量增加。黑色能量曲线显示,随着还原分子向电极靠近,吉布斯能量增加,
这两条能量曲线在$\Delta G^{*}(0)$相交。
对电极施加一个电势$E$将会使能量曲线下移\footnote{
    将离子的电位从$0$增加到$E$将会使得它们的$\Delta G$增加$E\Delta q$
    ,$\Delta q$为离子上的电荷。增加电极电位将降低电极附近离子相对于电极的电位,
    从而降低它们的$\Delta G$
}
(至红色曲线),交点会移动至$\Delta G^{*}(E)$。
$\Delta ^{\ddagger }G_{c}$和$\Delta ^{\ddagger }G_{a}$是施加电势$E$后,
氧化态物质与还原态物质反应时需要克服的活化能\footnote{又称势垒(Energy Barrier)}。$\Delta ^{\ddagger }G_{oc}$和$\Delta ^{\ddagger }G_{oa}$
则是没有施加电势$E$时氧化态物质与还原态物质反应时需要克服的活化能\cite{CITEREFNewmanThomas-Alyea2004}。
\begin{figure}[H]
    \centering
    \includegraphics[width=13cm]{ButlerVolmerGibbsPlot.png}\\
    各物质的吉布斯自由能与反应坐标关系图。反应将朝着降低能量方向进行\\
    蓝色曲线为还原,红色曲线为氧化。绿色曲线说明了反应达到平衡。
    \caption{吉布斯自由能$\Delta G$与反应坐标$\xi$及电极电势$E$的关系\cite{Gibbs}}
    \label{fig:ButlerVolmerGibbsPlot}
\end{figure}

假设反应的速率常数由式~\ref{equation:Arrhenius}Arrhenius方程确定:
\begin{equation}
    \begin{aligned}
        k_ {f} = A_ {f} \ exp [-\Delta ^ {\ddagger} G_ {c} / RT]\\
        k_ {b} = A_ {b} \ exp [-\Delta ^ {\ddagger} G_ {a} / RT]%&是用于标注需要对齐的位置
    \end{aligned}
\end{equation}

其中$A_ {f}$与$A_ {b}$是与反应速度有关的常数。假设能量曲线在过渡区域中几乎是线性的,则可以用以下方式表示:
\begin{equation}
    \begin{aligned}
        &\Delta G = S_{c} \xi + K_{c} & \mbox{蓝色曲线}\\
        &\Delta G = S_{c} \xi + K_{c} -nEF \quad &\mbox{红色曲线}\\
        &\Delta G =- S_{a} \xi + K_{a} \quad &\mbox{黑色曲线}
    \end{aligned}
\end{equation}

这种简单情况下的电荷转移系数等效于对称系数,可以用能量曲线的斜率表示:
\begin{equation}
    \alpha=\frac{S_ {c}}{S_ {a} + S_ {c}}
\end{equation}

推导出:
\begin{equation}
    \begin{aligned}
        &\Delta ^{\ddagger }G_{c}=\Delta ^{\ddagger }G_{oc}+\alpha nFE\\
        &\Delta ^{\ddagger }G_{a}=\Delta ^{\ddagger }G_{oa}-(1-\alpha )nFE
    \end{aligned}
\end{equation}

为简洁起见,我们定义:
\begin{equation}
    \begin{aligned}
        &f_{\alpha }=\alpha nF/RT\\
        &f_{\beta }=(1-\alpha )nF/RT\\
        &f=f_{\alpha }+f_{\beta }=nF/RT
    \end{aligned}
\end{equation}

速率常数现在可以表示为:
\begin{equation}
    \begin{aligned}
        &k_{f}=k_{fo}e^{-f_{\alpha }E}\\
        &k_{b}=k_{bo}e^{f_{\beta }E}
    \end{aligned}
\end{equation}

其中零电位下的速率常数为:
\begin{equation}
    \begin{aligned}
        &k_{fo}=A_{f}e^{-\Delta ^{\ddagger }G_{oc}/RT}\\
        &k_{bo}=A_{b}e^{-\Delta ^{\ddagger }G_{oa}/RT}
    \end{aligned}
\end{equation}

现在我们可以将电流密度$j$写作外加电势$E$的函数\cite{CITEREFNewmanThomas-Alyea2004}:
\begin{equation}
    j=nF(c_{r}k_{bo}e^{f_{\beta }E}-c_{o}k_{fo}e^{-f_{\alpha }E})
\end{equation}

在一定的电压$E_{eq}$下,反应达到平衡,$vf$和$vb$相等
(如图~\ref{fig:ButlerVolmerGibbsPlot}绿色曲线表示)。
此时平衡电流是相等的,写成$j_o$,也就是交换电流密度。
我们定义过电位$\eta=E-E_ {eq}$,此时电流密度$j$可以表达为:
\begin{equation}
    j=j_{0}\cdot \left\{\exp \left[{\frac {\alpha _{a}zF\eta }{RT}}\right]-\exp \left[-{\frac {\alpha _{c}zF\eta }{RT}}\right]\right\}
    \label{equation:Butler_Volmer}
\end{equation}

这被称为Butler-Volmer方程,描述了在一个电极上的施加电势与电极电流的关系。            %释放过程介绍
% !TEX root = ../main.tex
\chapter{建模与仿真}

\section{求解过程的简述}
根据第二章中推导的反应过程,仿真的过程主要分为以下三步:

1)在COMSOL Multiphysics中,对通电时发生的电化学反应进行建模,得到阴极极板表面的平均$\ce{OH-}$浓度随
时间的变化曲线。

2)利用第1步得到的浓度变化随时间的变化曲线,在MATLAB中计算$\ce{Zr^4+}$的水解程度随时间的变化关系,
再根据$\ce{Zr^4+}$消耗与DNA释放这一正比关系,得到DNA的释放曲线。

3)利用第2步得到的数据,在COMSOL Multiphysics中,对DNA在电场力以及浓度梯度力作用下的扩散、迁移过程进行建模仿真。
\section{反应容器物理模型}
本课题组使用的反应容器为一个长度为$2cm$,直径为$1cm$的圆柱体。作为电极的金薄膜的规格为$20mm×6mm×1mm$。

如图~\ref{fig:container} 所示,根据上述数据,我们可以在COMSOL Multiphysics中完成对反应溶液的物理模型的建立。
由于电极材料为金,不参与反应,内部也考虑为等势体,所以我们只用考虑在溶液体系内发生的反应,在建模时只用对溶液
进行建模。
\begin{figure}[ht]
    \centering
    \includegraphics[scale=1]{container.jpg}\\
    1)阳极(Anode) 2)阴极(Cathode) 3)电解液(Electrolyte)
    \caption{反应容器物理模型}
    \label{fig:container}
\end{figure}

\section{电化学反应仿真}
\subsection{计算流程}
——介绍COMSOL Multiphysics的操作及计算过程
\subsection{结果与讨论}
——讨论结果以及参数影响等

\section{Zr$^{4+}$水解反应仿真}
\subsection{计算流程}
——介绍在MATLAB中计算水解量
\subsection{结果与讨论}
——讨论结果以及参数影响等

\section{DNA扩散过程仿真}
\subsection{计算流程}
——介绍COMSOL Multiphysics的操作及计算过程
\subsection{结果与讨论}
——讨论结果以及参数影响等             %仿真
% !TEX root = ../main.tex
\chapter{系统特征的研究与系统优化}

\section{激励电压与激励时间的影响}
\subsection{激励电压的影响}

首先我们讨论通电后发生的电化学反应,在COMSOL Multiphysics继续使用第三章的电化学模型。
对激励电压进行参数扫描,得到图 ~\ref{fig:cOH_U}。可以分析得到激励时间一定的情况下,
激励电压越高,$\ce{OH-}$浓度的峰值越高,相应的在去除激励之后,$\ce{OH-}$浓度下降也越快。
使用高电压信号可以降低信道间干扰。

\begin{figure}[H]
    \centering
    \includegraphics[width=13cm]{不同电压激励下cOH平均浓度曲线.png}
    \caption{不同电压激励下阴极表面$\ce{OH-}$浓度随时间的变化曲线}
    \label{fig:cOH_U}
\end{figure}

定义集中系数$C$为$\ce{OH-}$浓度曲线峰值的4次方除以一个信号周期内
$\ce{OH-}$浓度的4次方的积分值\footnote{
    由反应速率方程可知,水解反应的反应速率与$\ce{OH-}$浓度的4次方成正比,
    所以这里用$c_{\ce{OH-}}^4$作为参数。
}:
\begin{equation}
    C=\frac{c_{\ce{OH-},max}^4}{\int_0^T{c_{\ce{OH-}}^4}dt}
\end{equation}

集中系数$C$反映了$\ce{OH-}$浓度曲线峰值相对于整个周期内$\ce{OH-}$浓度曲线的大小。

在激励结束后滞后的$c_{\ce{OH-}}$曲线造成水解反应持续发生,对
DNA浓度观测值造成的干扰。集中系数$C$越高,
激励电压施加时发生的水解反应越占主导地位,反映了该系统的抗码间干扰能力。

由表~\ref{tab:1}可知,
随着激励电压的增加,该系统的集中系数$C$变化不大,处于缓慢的震荡减少状态。
所以我们认为激励电压的大小对码间干扰的影响不大,随着激励电压升高,码间干扰会
有轻微增加,但不影响其性能表现。

\begin{table}

    % table caption is above the table
    
    \caption{不同电压激励下集中系数$C$}
    
    \label{tab:1}       % Give a unique label
    
    % For LaTeX tables use
    \centering
    \begin{tabular}{ll}
    
    \hline\noalign{\smallskip}
    
    电压($V$) & 集中系数 \\
    
    \noalign{\smallskip}\hline\noalign{\smallskip}
    
        \quad 1&\quad 0.1800  \\
        \quad 2&\quad 0.1741  \\
        \quad 3&\quad 0.1529  \\
        \quad 4&\quad 0.1632  \\
        \quad 5&\quad 0.1518  \\
    
    \noalign{\smallskip}\hline
    
    \end{tabular}
    
\end{table}

但本文未考虑电流的热效应影响,在高电压激励下,电解产生的热量会对系统造成影响,甚至破坏DNA结构,造成信息损失,
所以系统有一个最高电压,激励电压超过了该值会对系统的稳定性产生负面的影响。

综上所述,在安全电压范围内,激励电压越高,系统的抗信道间干扰能力越强。

\subsection{激励时长的影响}

我们考虑不同持续时间的5$V$激励电压信号对电解过程的影响,在COMSOL Multiphysics中对
激励时长进行参数扫描,得到结果如图~\ref{fig:cOH_t}所示,长时间的激励可以使$\ce{OH-}$
浓度处于一个较高水平,这将增加DNA的释放量,对抗信道间干扰有利。
\begin{figure}[H]
    \centering
    \includegraphics[width=13cm]{不同时间信号下cOH平均浓度曲线_2.png}
    \caption{不同激励时长下阴极表面$\ce{OH-}$浓度随时间的变化曲线}
    \label{fig:cOH_t}
\end{figure}

接着计算不同激励时长下的集中系数$C$,得到的结果如图~\ref{集中系数与时长}所示,
持续时间长的激励信号,其集中系数较低,这会导致其抗码间干扰能力下降,不能持续长时间
发送信号。
\begin{figure}[H]
    \centering
    \includegraphics[width=13cm]{集中系数与时长.png}
    \caption{集中系数与激励时长的关系}
    \label{集中系数与时长}
\end{figure}

综合抗码间干扰与抗信道间干扰的能力看,激励信号应有持续时间较短、激励电压较高的特性,高电压的脉冲函数是一个理想的信号源。

\section{接收机位置对最大传输速率的关影响}
\subsection{观测点与激励信号对DNA浓度曲线的影响}

在COMSOL Multiphysics中进行电泳的仿真后,我们修改探针的位置,得到不同位置的探针的DNA浓度曲线。
激励电压我们对比原始数据的5$V$10$s$激励与5$V$1$s$的激励。

如图~\ref{不同激励探针DNA曲线}所示,
距离阴极越近的位置,DNA浓度达到峰值的时间越短,峰值强度越大。在距离
超过10$cm$之后,由于扩散速度受限,在下一次激励之前,DNA浓度不能达到峰值。所以可以得出结论
该系统通信距离越短,通信效果越好。


对比10$s$时长信号与1$s$时长信号激励下的DNA浓度曲线,可以发现在短时长信号的激励下,相同位置处,
DNA浓度峰值会更早到达,峰值相对于整条曲线也会更突出。这也佐证了上一节我们提到的,一个激励时长较短
的信号能更加有效的抑制干扰这一观点。我们也可以得出激励时长较短的信号有助于提高通信频率这一结论。
\begin{figure}[H]
    \centering
    \subcaptionbox{5$V$10$s$激励下不同位置探针的DNA浓度曲线}% 
                    [14cm]{\includegraphics[width=13cm]{cDNA与t与y.png}}\\
    \subcaptionbox{5$V$1$s$激励下不同位置探针的DNA浓度曲线}% 
                    [14cm]{\includegraphics[width=13cm]{cDNA与t与y_1.png}}\\
    \caption{不同激励信号下不同位置探针的DNA浓度曲线}
    \label{不同激励探针DNA曲线}
\end{figure}

\subsection{不同观测点的最大通信频率}

我们定义DNA浓度在单次激励中的的上升时间$T_s$为其达到该信号周期
内浓度最大值$c_{max}$的$\frac{1}{\sqrt{2}}$时,
所用的时间。图~\ref{rising time}显示了在激励电压为5$V$,激励时长为1$s$的信号作用下,
距离阴极不同位置的观测点,DNA浓度的上升时间及其倒数。

\begin{figure}[H]
    \centering
    \subcaptionbox{离阴极不同距离的观测点处DNA浓度的上升时间}% 
                    [14cm]{\includegraphics[width=13cm]{raising time.png}}\\
    \subcaptionbox{离阴极不同距离的观测点处DNA浓度的上升时间的倒数}% 
                    [14cm]{\includegraphics[width=13cm]{raising_time_frequency.png}}\\
    \caption{离阴极不同距离的观测点处DNA浓度的上升时间及其倒数}
    \label{rising time}
\end{figure}

我们可以将达到浓度最大值$c_{max}$的$\frac{1}{\sqrt{2}}$时的时刻作为采样点,这时
最大通信频率为上升时间的倒数。图~\ref{rising time}的数据显示,在距离阴极0.5mm的地方,
我们可以达到0.909Hz的最大通信频率;在距离阴极4.5mm处,我们可以达到0.2778Hz的最大通信频率,
而在距离阴极9.5mm处,最大通信频率只有0.05348Hz。

数据说明了,基于自由扩散的通信系统,不适合远距离通信。接收机放置在距发送机5$mm$这一范围内时,
能达到一个较快的通信速度,而且抗干扰能力强。在超过7$mm$这一范围后,通讯的间隔必须增加到数十秒,
以抑制码间干扰,而且其抗信道间干扰能力也会下降。

\section{系统阈值}
\subsection{基于水解反应的$c_{\ce{OH-}}$阈值}
由$\ce{Zr^4+}$的水解方程~\ref{Zr4+水解}可以得知,
$\ce{Zr^4+}$参与水解的条件应该为:
\begin{equation}
    \frac{c_{\ce{H+}}^4}{c_{\ce{Zr^4+}}  }<K°=10^{1.9}
\end{equation}

由水溶液中存在水的电离,将$\ce{H+}$与$\ce{OH-}$的浓度相联系,可以将上述式子变形为:
\begin{equation}
    \frac{k_w^4}{c_{\ce{Zr^4+}} c_{\ce{OH-}}^4 }<K°=10^{1.9}
\end{equation}

即当$c_{\ce{OH-}}$满足如下条件时,水解反应才会开始:
\begin{equation}
    c_{\ce{OH-}}>\frac{k_w}{(c_{\ce{Zr^4+}}K°)^{1/4}}
\end{equation}

其中$K°=10^{1.9}$,$K_w=10^{-14}$。考虑$\ce{Zr^4+}$在LbL结构中的离子间间距为3~4$nm$
\cite{\cite{Shervedani2011Electrochemical}},由此计算其在阴极表面的活度
\footnote{又称有效浓度,有效摩尔分数}:
\begin{equation}
    c_{\ce{Zr^4+}}=(\frac{1dm}{3.5nm})^3\cdot \frac{1}{N_A} mM
    =0.0387mM
\end{equation}

其中$N_A$为阿伏伽德罗常量,取$6.022\times 10^{23}$。计算得到$c_{\ce{OH-}}$的阈值为
$7.55\times 10^{-15}mM$,该水解反应启动的浓度阈值非常小,可以认为在$\ce{OH-}$存在时,反应
都会发生,由于反应速度与$c_{\ce{OH-}}$的4次方成正比,所以低浓度时,反应速度极慢,在所观测的
时间尺度下,可以认为不施加激励电压时反应不发生。
\subsection{基于电化学反应的电压阈值}

在式~\ref{equation:Butler_Volmer}中我们提到了,在电化学反应中,施加的电压必须超过平衡电压,
即过电位应大于0时,反应才会正向发生。引用\parencite{C9RP00218A}一文的介绍,对于本文中的溶液环境,
理论上电化学反应的开启电压为1.229$V$,考虑式~\ref{equation:cathode_reaction}中水直接参与
电极反应所需要的能量,该电化学反应的电压阈值应为2.17$V$,但是在实际条件下,
由于电极阻抗、溶液电导率等因素的影响,可能需要更大的电压来启动此反应\footnote{
    https://chemed.chem.purdue.edu/genchem/topicreview/bp/ch20/faraday.php\#aq
},如工业上电解食盐水的实际电压在3.5~4.0$V$。

\subsubsection{DNA/$\ce{Zr^4+}$单层的释放次数}
在激励电压为5$V$激励时长为1$s$的信号作用下,$\ce{Zr^4+}$的消耗曲线如图~\ref{5_1_Zr曲线}所示。由于
$\ce{Zr^4+}$的消耗,DNA也被释放进入溶液中,LbL结构分解。我们可以得到在该激励信号的作业下,每次激励
消耗的$\ce{Zr^4+}$/DNA单层会被消耗16\%。所以理论上,在高电压、短时长激励信号的作用下,
每一层$\ce{Zr^4+}$/DNA单层最多可以释放6次。

\begin{figure}[H]
    \centering
    \includegraphics[width=8cm]{Zr阈值.png}
    \caption{5$V$1$s$激励下$\ce{Zr^4+}$消耗曲线}
    \label{5_1_Zr曲线}
\end{figure}



             %系统特征的研究与系统优化
% !TEX root = ../main.tex

\begin{summary}
这里是全文总结内容。
近年来,分子通信研究高速发展,诞生了很多新理论与新技术。
一种基于DNA/$\ce{Zr^{4+}}$层层自组装结构在电信号控制下分解的技术
,实现了DNA的可控释放。本文使用COMSOL Multiphysics与MATLAB软件,
在课题组实验数据基础上,综合电化学反应原理、化学动力学原理、扩散原理
等对DNA的受控释放过程进行了建模与仿真,重点对激励信号与接收机位置
进行了研究,分析和总结了影响该系统通信速度、抗干扰能力、总发送量
这三个因素的影响。本文主要研究结果如下:

1)通过对分子通信的学习,了解了目前分子的发展现状:目前分子通信的系统领域已经有多种概念模型和基本框架被建立,
但在如何提高纳米机器的性能水平,如何更有效地传输信号分子,如何提升通信距离以及抗干扰能力这三个方面,
仍有很多挑战等待科研人员,

2)通过对有关化学理论的查阅与理解,本文提出了一种DNA/$\ce{Zr^{4+}}$
层层自组装结构的
分解的原理。在电信号的激励下,阴极处会发生式~\ref{equation:cathode_reaction}
所描述的化学反应,这使得阴极处$\ce{OH-}$浓度增加,促使$\ce{Zr^4+}$
发生式~\ref{Zr4+水解}描述的水解反应,$\ce{Zr^4+}$的消耗导致了
DNA/$\ce{Zr^{4+}}$层层自组装结构的分解,被释放到溶液中的DNA在
浓度梯度力与电场力的共同作用下,在溶液中扩散。

3)在COMSOL Multiphysics与MATLAB上对上述反应过程进行数值仿真,
在引用课题组实验的模型、参数的情况下结果较好的还原了实验数据。
验证了电极反应、水解反应、电泳过程共同作用的仿真可以适用于对
该分子通信系统的研究。

4)不考虑电解热效应的情况下,激励电压越高,系统的抗信道间干扰能力越强。
激励时长越短,系统的抗码间干扰能力越强。
激励信号应有持续时间较短、激励电压较高的特性,
高电压的脉冲函数是一个理想的信号源。

5)距离阴极越近的位置,DNA浓度达到峰值的时间越短,峰值强度越大。
系统通信距离越短,通信效果越好。
短时长的信号的激励下,相同位置处,DNA浓度峰值会更早到达,
峰值相对于整条曲线也会更突出。
在距离阴极0.5mm的地方,
我们可以达到0.909Hz的最大通信频率;在距离阴极4.5mm处,我们可以达到0.2778Hz的最大通信频率,
而在距离阴极9.5mm处,最大通信频率只有0.05348Hz。
数据说明了,基于自由扩散的通信系统,不适合远距离通信。接收机放置在距发送机5$mm$这一范围内时,
能达到一个较快的通信速度,而且抗干扰能力强。在超过7$mm$这一范围后,通讯的间隔必须增加到数十秒,
以抑制码间干扰,而且其抗信道间干扰能力也会下降。

6)综合考虑平衡常数,在$\ce{OH-}$存在时,$\ce{Zr^4+}$水解反应
都会发生,由于反应速度与$c_{\ce{OH-}}$的4次方成正比,所以低浓度时,反应速度极慢,在所观测的
时间尺度下,可以认为不施加激励电压时反应不发生。
电化学反应的电压阈值应为2.17$V$,但实际条件下,
由于电极阻抗、溶液电导率等因素的影响,可能需要更大的电压来启动此反应。
通过对激励电压与时长的控制,一个DNA/$\ce{Zr^4+}$单层可以被多次释放,
在激励电压为5$V$激励时长为1$s$的信号作用下每一层$\ce{Zr^4+}$/DNA单层最多可以释放6次。


\end{summary}
                %结论与展望
%使用名词作为标题
%保证在黑白打印的情况下,图例分明
%图要大,看得清,坐标轴表示什么要标好。
%matlab下font至少20
%TC:ignore

% 使用英文字母对附录编号
\appendix

% 附录内容,本科学位论文可以用翻译的文献替代。
%\input{contents/app_maxwell_equations}
%\input{contents/app_flow_chart}

% 文后无编号部分
\backmatter

% 参考资料
\printbibliography[heading=bibintoc]

% 用于盲审的论文需隐去致谢、发表论文、参与项目、申请专利、简历

% 致谢
% !TEX root = ../main.tex

\begin{acknowledgements}
  在论文写作的全程,我的导师闫浩对我进行了悉心的指导,通过线上会议及答疑的方式严格按照论文写作规划和指导要求认真负责的与我沟通,对我不胜其烦的问题给予积极耐心的回应。老师提出了诸多宝贵意见和建议,让我对毕业论文写作有了更深入全面的了解。在她的帮助和鼓励下,我的研究思路得以拓宽,论作水平得到了很大的提高。同时,老师的严谨治学的作风和认真负责的态度对我在论文写作中也产生了莫大的积极影响,在老师的带领下我以高度的重视和严谨的态度认真完成论文写作。对闫浩老师一直以来的帮助与鼓励,再次向她表达我由衷的感谢。
  
  同时,也要感谢四年以来上海交通大学的每一位任课老师,正是基于他们往日严谨负责的教学,使我学有所获,积累相关专业知识,为本次的论文写作打下扎实的基础。在此表达我诚挚的谢意和祝愿。
  
  然后,也要对陪我共度四年大学时光的朋友们给予诚挚感谢。Mogami Shizuka在精神上给了我莫大鼓励,在无助的时候给予我前行的动力。鸦教练、呆呆等雀士与我在Mahjong部共度了许多美好的时光。偶研社的
  社员们给我留下了诸多深刻的回忆。在此也十分感谢一直他们的相伴与鼓励。最后特别感谢我的女朋友在4年多来对我的无私陪伴,同我度过最美好的青春时光。
  
\end{acknowledgements}


% 发表论文、参与项目、申请专利、简历
% 盲审论文中,发表学术论文及参与科研情况等仅以第几作者注明即可,不要出现作者或他人姓名
%\input{contents/publications}
%\input{contents/achievements}
%\input{contents/resume}

% 中文学士学位论文要求在最后有一个英文大摘要,单独编页码,英文学士学位论文不需要
% !TEX root = ../main.tex

\begin{digest}
  to do
\end{digest}


%TC:endignore

\end{document}

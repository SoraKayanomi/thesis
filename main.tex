% !TeX encoding = UTF-8

% 载入 SJTUThesis 模版
\documentclass[degree=bachelor, zihao=-4]{sjtuthesis}
% 选项
%   degree=[doctor|master|bachelor|course],   % 可选(默认:doctor),学位类型
%   zihao=[-4|5],                             % 可选(默认:5),正文字号大小
%   language=[chinese|english],               % 可选(默认:chinese),论文的主要语言
%   review,                                   % 可选(默认:关闭),盲审模式
%   [twoside|oneside]                         % 可选(默认:twoside),单双页模式

% 论文基本配置,加载宏包等全局配置
% !TEX root = ./main.tex

\sjtusetup{
  %
  %******************************
  % 注意:
  %   1. 配置里面不要出现空行
  %   2. 不需要的配置信息可以删除
  %******************************
  %
  % 信息录入
  %
  info = {%
    %
    % 标题
    %   可使用“\\”命令手动控制换行
    %
    title           = {DNA分子通信系统的系统模型研究},
    title*          = {A Study of Systems Modeling in DNA-Based Molecular Communications},
    %
    % 关键词
    %
    keywords        = {DNA信息分子发送器, 微观分子通信实验平台, 微观DNA分子通信建模},
    keywords*       = {DNA Information Molecule Transmitter, 
    Experimental Platform of Micro Molecular Communication, Micro DNA Molecular Communication Modeling},
    %
    % 姓名
    %
    author          = {孙文韬},
    author*         = {Wentao\quad{}Sun},
    %
    % 指导教师
    %
    supervisor      = {闫浩},
    supervisor*     = {Hao Yan},
    %
    % 副指导教师
    %
    % assisupervisor  = {某某教授},
    % assisupervisor* = {Prof. Uom Uom},
    %
    % 学号
    %
    id              = {516030910265},
    %
    % 学位
    %   本科生不需要填写
    %
    degree          = {工学学士},
    degree*         = {Bachelor of Engineering},
    %
    % 专业
    %
    major           = {测控技术与仪器},
    major*          = {Measurement and Control Technology and Instrument Science},
    %
    % 所属院系
    %
    department      = {仪器科学与工程系},
    department*     = {Depart of Instrument Science and Engineering},
    %
    % 课程名称
    %   仅课程论文适用
    %
    coursename      = {某某课程},
    %
    % 答辩日期
    %   使用 ISO 格式;默认为当前时间
    %
    % date            = {2014-12-17},
    %
    % 资助基金
    %
    % fund  = {
    %           {国家 973 项目 (No. 2025CB000000)},
    %           {国家自然科学基金 (No. 81120250000)},
    %         },
    % fund* = {
    %           {National Basic Research Program of China (Grant No. 2025CB000000)},
    %           {National Natural Science Foundation of China (Grant No. 81120250000)},
    %         },
  },
  %
  % 格式设置
  %
  format = {%
    %
    % 本科论文页眉 logo 颜色
    %   默认为黑色
    %
    % header-logo-color = red,
  },
  %
  % 名称设置
  %
  name = {%
    % publications      = {攻读学位期间完成的论文},
  },
}

% 参考文献支持宏包
\usepackage[backend=biber,style=gb7714-2015,gbpub=false,gbpunctin=false]{biblatex}
% 导入参考文献数据库
\addbibresource{bibdata/thesis.bib}

% 定义图片文件目录与扩展名
\graphicspath{{figures/}}
\DeclareGraphicsExtensions{.pdf,.eps,.png,.jpg,.jpeg}

% 确定浮动对象的位置,可以使用 [H],强制将浮动对象放到这里(可能效果很差)
\usepackage{float}

% 固定宽度的表格
\usepackage{tabularx}

% 表格中支持跨行
\usepackage{multirow}

% 表格中数字按小数点对齐
\usepackage{dcolumn}
\newcolumntype{d}[1]{D{.}{.}{#1}}

% 附带脚注的表格
\usepackage{threeparttable}

% 算法环境宏包
\usepackage[ruled,vlined,linesnumbered]{algorithm2e}
% \usepackage{algorithm}

% 代码环境宏包
\usepackage{listings}
\lstnewenvironment{codeblock}[1][]
  {\lstset{style=lstStyleCode,#1}}{}

% 国际单位制宏包
\usepackage{siunitx}

% 定理环境宏包
\usepackage{ntheorem}
% \usepackage{amsthm}

% 绘图宏包
\usepackage{tikz}

% 化学方程式宏包
\usepackage[version=4]{mhchem}

%tikz宏包
\usepackage{tikz}

% 一些文档中用到的 logo
\usepackage{hologo}
\newcommand{\XeTeX}{\hologo{XeTeX}}
\newcommand{\BibLaTeX}{\textsc{Bib}\LaTeX}

% 借用 ltxdoc 里面的几个命令。
\def\cmd#1{\cs{\expandafter\cmd@to@cs\string#1}}
\def\cmd@to@cs#1#2{\char\number`#2\relax}
\DeclareRobustCommand\cs[1]{\texttt{\char`\\#1}}

\newcommand*{\meta}[1]{{%
  \ensuremath{\langle}\rmfamily\itshape#1\/\ensuremath{\rangle}}}
\providecommand\marg[1]{%
  {\ttfamily\char`\{}\meta{#1}{\ttfamily\char`\}}}
\providecommand\oarg[1]{%
  {\ttfamily[}\meta{#1}{\ttfamily]}}
\providecommand\parg[1]{%
  {\ttfamily(}\meta{#1}{\ttfamily)}}
\providecommand\pkg[1]{{\sffamily#1}}

% 自定义命令

% E-mail
\newcommand{\email}[1]{\href{mailto:#1}{\texttt{#1}}}

% hyperref 宏包在最后调用
\usepackage{hyperref}


\begin{document}

%TC:ignore

% 无编号内容:中英文论文封面、授权页
\maketitlepage
\makeorigpage*
\makeauthpage[scans/authorization.pdf]

% 使用罗马数字对前言编号
\frontmatter

% 摘要
% !TEX root = ../main.tex

\begin{abstract}
  中文摘要应该将学位论文的内容要点简短明了地表达出来,应该包含论文中的基本信息,
  体现科研工作的核心思想。摘要内容应涉及本项科研工作的目的和意义、研究方法、研究
  成果、结论及意义。注意突出学位论文中具有创新性的成果和新见解的部分。摘要中不宜
  使用公式、化学结构式、图表和非公知公用的符号和术语,不标注引用文献编号。硕士学
  位论文中文摘要字数为 500 字左右,博士学位论文中文摘要字数为 800 字左右。英文摘
  要内容应与中文摘要内容一致。

  摘要页的下方注明本文的关键词(4~6个)。
\end{abstract}

\begin{abstract*}
  Shanghai Jiao Tong University (SJTU) is a key university in China. SJTU was
  founded in 1896. It is one of the oldest universities in China. The University
  has nurtured large numbers of outstanding figures include JIANG Zemin, DING
  Guangen, QIAN Xuesen, Wu Wenjun, WANG An, etc.

  SJTU has beautiful campuses, Bao Zhaolong Library, Various laboratories. It
  has been actively involved in international academic exchange programs. It is
  the center of CERNet in east China region, through computer networks, SJTU has
  faster and closer connection with the world.
\end{abstract*}


% 目录、插图目录、表格目录
\tableofcontents
\listoffigures*
\listoftables*
\listofalgorithms*

% 主要符号、缩略词对照表
% !TEX root = ../main.tex

\begin{nomenclature*}
\label{chap:symb}

\begin{longtable}{rl}
  $\epsilon$    & 介电常数 \\  
  $\mu$         & 磁导率 \\
\end{longtable}

\end{nomenclature*}


%TC:endignore

% 使用阿拉伯数字对正文编号
\mainmatter

% 正文内容
\input{contents/literature_review}      %文献综述
% !TEX root = ../main.tex
\chapter{}
\section{}               %模型建立
\input{contents/simulation_1}           %电化学反应仿真(电解食盐水)
\input{contents/simulation_2}           %反应动力学反应仿真(Zr4+水解)
\input{contents/simulation_3}           %DNA电泳仿真
% !TEX root = ../main.tex

\begin{summary}
针对分子通信领域缺少微观实验平台的瓶颈问题,
本课题组提出了以DNA为信息分子的微观分子通信实验平台
 ,并初步完成了其中关键部件的DNA信息分子的发送器与接收器的设计与实现
。然而,针对DNA分子通信,领域内并没有相关的理论研究。但是,
对实验平台的深入理解,离不开理论研究的指导。针对这个问题,
本毕业设计立足于课题组的现有的DNA分子通信实验平台,对其中DNA信息分子发送器部分开展理论研究。\\

首先,我们融合化学、物理学科的知识,利用COMSOL Multiphysics与MATLAB软件,基于电化学反应原理、化学动力学原理、粒子自由扩散与带电粒子在电场辅助下自由扩散的原理,获得了DNA信息分子发送器的三维理论模型。其次,我们利用获得的DNA信息分子发送器模型,研究了关键参数对DNA发送过程的影响,包括:激励信号与接收机位置等,并分析了关键系统参数对信息传播速度、抗干扰能力、总发送量这三个性能的影响。通过本工作的研究,课题组深入理解了DNA信息分子发送器的特征,指明了优化DNA信息分子发送器的具体方向。对进一步推进本课题组的微观分子通信实验平台的研究具有重要意义。\\

本工作的主要研究结果包含如下几个方面:

1)通过学习,了解了分子通信领域的发展现状与分子通信的基本原理。 

2)通过调研与阅读多学科文献,学习、归纳并总结了DNA信息分子发送器中关键的多层DNA结构的组装过程
、理化性质以及DNA分子的扩散运动运动过程。通过自己的理解与调研
,详细分析与讨论了DNA信息分子分解与释放过程中的电化学反应、水解反应、
电迁移及扩散过程,并最终提出DNA/$\ce{Zr^4+}$多层结构的分解与释放的原理。并
将上述多个过程表达为对应的理论方程。

3)利用两个软件,COMSOL Multiphysics 与 MATLAB,对上述DNA信息分子发送过
程进行数值仿真,仿真结果与课题组的实验数据符合度好,证明了所提出的模型的准确性与有效性。
也验证了利用COMSOL Multiphysics 与 MATLAB对电极反应、水解反应、
电泳等多个过程共同作用的仿真可以用于对DNA分子通信系统的建模。

4)对DNA信息分子发送器的系统特征进行研究,深入理解了DNA信息分子发送器的
特征,指明了优化DNA信息分子发送器的具体方向。关键结论包括:

$\bullet$讨论了激励电压信号的强度与持续时间对DNA信息分子发送过程的影响。不考虑电解热效应的情况下,
激励电压越高,系统抗码间串扰的能力越强。激励时长越短,系统抗码间串扰的能力越强。激励信号需要具备持续时间较短
、激励电压较高的特性,高电压的脉冲函数是一个理想的信号源。

$\bullet$讨论了接收机位置对系统性能的影响,获得信道中不同位置初的DNA浓度变化曲线并研究了DNA信息分子发送器
的阈值。接收机距离阴极越近,DNA浓度达到峰值的时间越短,峰值强度越大。接收机距离越短,信号传输效果越好。
短时长的信号的激励下,相同位置处,DNA浓度峰值会更早到达,峰值相对于整条曲线也会更突出。
在距离阴极 0.5$mm$的地方,我们可以达到54.5$bit/min$的最大通信速率;
在距离阴极 4.5$mm$处,我们可以达到16.7$bit/min$的最大通信频率,而在距离阴极9.5$mm$处,最大通信频率只有3.2$bit/min$。
数据说明了,基于自由扩散的通信系统,
不适合远距离通信。接收机放置在距发送机5$mm$以内这一范围内时,能达到一个较快的传输速度,
而且抗干扰能力强。在超过7$mm$这一范围后,
信号间隔必须增加到数十秒,以抑制码间串扰,而且其抗码间串扰的能力能力也会下降。

$\bullet$综合考虑平衡常数,在$\ce{OH-}$存在时,$\ce{Zr^4+}$水解反应都会发生,
由于反应速度与$c_{\ce{OH-}}$的4次方成正比,所以低$\ce{OH-}$浓度时,
反应速度极慢,在所观测的时间尺度下,可以认为不施加激励电压时反应不发生。
电化学反应的电压阈值应为2.17$V$。但实际条件下,由于电极阻抗、溶液电导率等因素的影响,
可能需要更大的电压来启动此反应。通过对激励电压与时长的控制,一个DNA/$\ce{Zr^4+}$单层可以被多次释放,
在激励电压为5$V$,激励时长为1$s$的信号作用下,每一层DNA/$\ce{Zr^4+}$单层最多可以释放6次。


\end{summary}
                %结论与展望

%TC:ignore

% 使用英文字母对附录编号
\appendix

% 附录内容,本科学位论文可以用翻译的文献替代。
% !TEX root = ../main.tex

\chapter{Maxwell Equations}

选择二维情况,有如下的偏振矢量:
\begin{subequations}
  \begin{align}
    {\bf E} &= E_z(r, \theta) \hat{\bf z}, \\
    {\bf H} &= H_r(r, \theta) \hat{\bf r} + H_\theta(r, \theta) \hat{\bm\theta}.
  \end{align}
\end{subequations}
对上式求旋度:
\begin{subequations}
  \begin{align}
    \nabla \times {\bf E} &= \frac{1}{r} \frac{\partial E_z}{\partial\theta}
      \hat{\bf r} - \frac{\partial E_z}{\partial r} \hat{\bm\theta}, \\
    \nabla \times {\bf H} &= \left[\frac{1}{r} \frac{\partial}{\partial r}
      (r H_\theta) - \frac{1}{r} \frac{\partial H_r}{\partial\theta} \right]
      \hat{\bf z}.
  \end{align}
\end{subequations}
因为在柱坐标系下,$\overline{\overline\mu}$ 是对角的,所以 Maxwell 方程组中电场
$\bf E$ 的旋度:
\begin{subequations}
  \begin{align}
    & \nabla \times {\bf E} = \upi \omega {\bf B}, \\
    & \frac{1}{r} \frac{\partial E_z}{\partial\theta} \hat{\bf r} -
      \frac{\partial E_z}{\partial r}\hat{\bm\theta} = \upi \omega \mu_r H_r
      \hat{\bf r} + \upi \omega \mu_\theta H_\theta \hat{\bm\theta}.
  \end{align}
\end{subequations}
所以 $\bf H$ 的各个分量可以写为:
\begin{subequations}
  \begin{align}
    H_r &= \frac{1}{\upi \omega \mu_r} \frac{1}{r}
      \frac{\partial E_z}{\partial\theta}, \\
    H_\theta &= -\frac{1}{\upi \omega \mu_\theta}
      \frac{\partial E_z}{\partial r}.
  \end{align}
\end{subequations}
同样地,在柱坐标系下,$\overline{\overline\epsilon}$ 是对角的,所以 Maxwell 方程
组中磁场 $\bf H$ 的旋度:
\begin{subequations}
  \begin{align}
    & \nabla \times {\bf H} = -\upi \omega {\bf D}, \\
    & \left[\frac{1}{r} \frac{\partial}{\partial r}(r H_\theta) - \frac{1}{r}
      \frac{\partial H_r}{\partial\theta} \right] \hat{\bf z} = -\upi \omega
      {\overline{\overline\epsilon}} {\bf E} = -\upi \omega \epsilon_z E_z
      \hat{\bf z}, \\
    & \frac{1}{r} \frac{\partial}{\partial r}(r H_\theta) - \frac{1}{r}
      \frac{\partial H_r}{\partial\theta} = -\upi \omega \epsilon_z E_z.
  \end{align}
\end{subequations}
由此我们可以得到关于 $E_z$ 的波函数方程:
\begin{equation}
  \frac{1}{\mu_\theta \epsilon_z} \frac{1}{r} \frac{\partial}{\partial r}
  \left(r \frac{\partial E_z}{\partial r} \right) + \frac{1}{\mu_r \epsilon_z}
  \frac{1}{r^2} \frac{\partial^2E_z}{\partial\theta^2} +\omega^2 E_z = 0.
\end{equation}

% !TEX root = ../main.tex

\chapter{绘制流程图}

图~\ref{fig:flow_chart} 是一张流程图示意。使用 \pkg{tikz} 环境,搭配四种预定义节
点(\verb+startstop+、\verb+process+、\verb+decision+和\verb+io+),可以容易地绘
制出流程图。

\begin{figure}[!htp]
  \centering
  \resizebox{6cm}{!}{\begin{tikzpicture}[node distance=2cm]
    \node (pic) [startstop] {待测图片};
    \node (bg) [io, below of=pic] {读取背景};
    \node (pair) [process, below of=bg] {匹配特征点对};
    \node (threshold) [decision, below of=pair, yshift=-0.5cm] {多于阈值};
    \node (clear) [decision, right of=threshold, xshift=3cm] {清晰?};
    \node (capture) [process, right of=pair, xshift=3cm, yshift=0.5cm] {重采};
    \node (matrix_p) [process, below of=threshold, yshift=-0.8cm] {透视变换矩阵};
    \node (matrix_a) [process, right of=matrix_p, xshift=3cm] {仿射变换矩阵};
    \node (reg) [process, below of=matrix_p] {图像修正};
    \node (return) [startstop, below of=reg] {配准结果};
     
    %连接具体形状
    \draw [arrow](pic) -- (bg);
    \draw [arrow](bg) -- (pair);
    \draw [arrow](pair) -- (threshold);

    \draw [arrow](threshold) -- node[anchor=south] {否} (clear);

    \draw [arrow](clear) -- node[anchor=west] {否} (capture);
    \draw [arrow](capture) |- (pic);
    \draw [arrow](clear) -- node[anchor=west] {是} (matrix_a);
    \draw [arrow](matrix_a) |- (reg);

    \draw [arrow](threshold) -- node[anchor=east] {是} (matrix_p);
    \draw [arrow](matrix_p) -- (reg);
    \draw [arrow](reg) -- (return);
\end{tikzpicture}
}
  \bicaption{绘制流程图效果}{Flow chart}
  \label{fig:flow_chart}
\end{figure}


% 文后无编号部分
\backmatter

% 参考资料
\printbibliography[heading=bibintoc]

% 用于盲审的论文需隐去致谢、发表论文、参与项目、申请专利、简历

% 致谢
% !TEX root = ../main.tex

\begin{acknowledgements}
  感谢那位最先制作出博士学位论文 \LaTeX 模板的交大物理系同学!

  感谢 William Wang 同学对模板移植做出的巨大贡献!

  感谢 \href{https://github.com/weijianwen}{@weijianwen} 学长一直以来的开发和维
  护工作!

  感谢 \href{https://github.com/sjtug}{@sjtug} 以及
   \href{https://github.com/dyweb}{@dyweb} 对 0.9.5 之后版本的开发和维护工作!

  感谢所有为模板贡献过代码的同学们, 以及所有测试和使用模板的各位同学!

  感谢 \LaTeX 和 \href{https://github.com/sjtug/SJTUThesis}{\sjtuthesis},帮我节
  省了不少时间。
\end{acknowledgements}


% 发表论文、参与项目、申请专利、简历
% 盲审论文中,发表学术论文及参与科研情况等仅以第几作者注明即可,不要出现作者或他人姓名
%% !TEX root = ../main.tex

\begin{publications}
  \item Chen H, Chan C~T. Acoustic cloaking in three dimensions using acoustic metamaterials[J]. Applied Physics Letters, 2007, 91:183518.
  \item Chen H, Wu B~I, Zhang B, et al. Electromagnetic Wave Interactions with a Metamaterial Cloak[J]. Physical Review Letters, 2007, 99(6):63903.
\end{publications}

\begin{publications*}
  \item 第一作者. 中文核心期刊论文, 2007.
  \item 第一作者. EI 国际会议论文, 2006.
\end{publications*}

%% !TEX root = ../main.tex

\begin{achievements}
  \item 第一发明人,“永动机”,专利申请号202510149890.0
\end{achievements}

\begin{achievements*}
  \item 第一发明人,“永动机”,专利申请号XXXXXXXXXXXX.X
\end{achievements*}

%% !TEX root = ../main.tex

\begin{resume}
  \subsection*{基本情况}
    某某,yyyy 年 mm 月生于 xxxx。

  \subsection*{教育背景}
  \begin{itemize}
    \item yyyy 年 mm 月至今,上海交通大学,博士研究生,xx 专业
    \item yyyy 年 mm 月至 yyyy 年 mm 月,上海交通大学,硕士研究生,xx 专业
    \item yyyy 年 mm 月至 yyyy 年 mm 月,上海交通大学,本科,xx 专业
  \end{itemize}

  \subsection*{研究兴趣}
    \LaTeX{} 排版

  \subsection*{联系方式}
  \begin{itemize}
    \item 地址: 上海市闵行区东川路 800 号,200240
    \item E-mail: \email{xxx@sjtu.edu.cn}
  \end{itemize}
\end{resume}


% 中文学士学位论文要求在最后有一个英文大摘要,单独编页码,英文学士学位论文不需要
% !TEX root = ../main.tex

\begin{digest}
  Molecular communication is an important communication scheme to establish a nano-machine network, which can solve the current problem that the limited capacity of a single nanomachine makes it difficult to carry out any concrete applications. The nano-network built by molecular communication technology can enable the entire nano-machine group to perform more complex tasks in a wider range, which is beneficial to the development of nano-machine applications, such as micro-surgery through the cooperation of a large number of nano-machines or drug delivery inside the human body. Therefore, molecular communication is a promising research direction and has received extensive attention.\\

  At present, the main research work in the field of molecular communication is concentrated on theory. So far, there is no complete microscopic molecular communication experimental platform, which seriously limits the development of the field of molecular communication and becomes a bottleneck in this field. Aiming at the bottleneck in this field, our group proposed a micro-molecular communication experiment platform using DNA as the information molecule, and completed the transmitter and receiver of DNA information molecules, which are the key components of the experiment platform. However, there is no relevant theoretical research on DNA molecular communication in this field. However, a thorough understanding of the experimental platform is inseparable from the guidance of theoretical research.\\
  
  To solve the above problems, this work is based on the existing DNA molecular communication experimental platform of our group and carries out theoretical research on its transmitter part. The transmitter of DNA information molecules that we implemented is a conversion interface from macroscopic electrical signals to microscopic DNA signals. In this contribution we illustrate a simple layer-by-layer assembly strategy to immobilize DNA on the surface of gold thin film and to trigger the selective release of DNA upon the exclusive control ofexternal electric potential. The assembly of DNA-containing multilayer films is driven by coordination/electrostatic interactions between inorganic zirconium ion (Zr4+) and phosphate groups in the backbone of the DNA chain. Its main principle is to use zirconium ion Zr4+ as the adhesive glue, and use the layer-by-layer self-assembly technology to fix the multilayer DNA/Zr4+ structure on the surface of the gold film. The decomposition and controllable release of multilayer DNA structures are performed through electrochemical reactions. This work carries out complete mathematical modeling on the above DNA decomposition and release process. Specifically, we use the electrochemical reaction model, chemical kinetic model and particle free diffusion model in physics to study the overall process of decomposition, release and propagation of the DNA multilayer structure. Using two softwares, COMSOL Multiphysics and MATLAB, this process was studied and a three-dimensional model of DNA decomposition and release process was established. Using the obtained model, the influence of key parameters on the process of DNA decomposition and release was studied, and then the characteristics of the DNA information molecular transmitter were deeply understood, and the specific direction of optimizing the DNA information molecular transmitter was pointed out. It is of great significance to further promote the research of the micro-molecular communication experiment platform in our group.\\
  
  The main work of this work includes the following aspects:
  
  1) A brief introduction to the basic principles and development status of molecular communication. The basic principles of the micro-molecular communication experiment platform of our group were introduced, and the research content and significance of the graduation project work were clarified. The application example of the key software COMSOL Multiphysics of this work in the field of molecular communication is introduced.

  2) The assembly process and physical and chemical properties of the key multi-layer DNA structure in the DNA information molecular transmitter are introduced. The electrochemical reaction, hydrolysis reaction, electromigration and diffusion process during the decomposition and release of DNA information molecules are analyzed and discussed in detail. The theoretical equations expressing the above process and particle motion process are introduced and explained in detail.
  
  3) Using COMSOL Multiphysics, three-dimensional modeling and simulation were carried out on the electrochemical reaction, hydrolysis reaction, electromigration and diffusion process of DNA information molecule decomposition, release and propagation. The obtained DNA concentration curve data from our established model agrees with the previous experimental data. Thus correctness of the obtained model of our DNA information molecular transmitter is validated.
  
  4) Study the system characteristics of DNA information molecular transmitter. The influence of the intensity and duration of the excitation voltage signal on the transmission process of DNA information molecules is discussed, and the DNA concentration change curves at different positions in the channel are obtained and the threshold of the DNA information molecule transmitter is studied. In-depth understanding of the characteristics of DNA information molecular transmitter is achieved which points out the specific direction of optimization of our DNA information molecular transmitter,including:
  
  $\bullet$The influence of the intensity and duration of the excitation voltage signal on the transmission process of DNA information molecules is discussed. Without considering the thermal effect of electrolysis, the higher the excitation voltage is, the stronger the ability of the system to resist the intersymbol interference will be. The shorter the excitation time is, the stronger the system's ability to resist the intersymbol interference will be. The excitation signal needs to have the characteristics of short duration and high excitation voltage, thus a high voltage pulse function is an ideal signal source.
  
  $\bullet$The influence of receiver position on system performance is discussed. The closer the receiver is to the cathode, the shorter the time for DNA concentration to reach the peak value, and the greater the peak intensity. The shorter the receiver distance, the better the signal transmission effect. Under the stimulation of short-term and long-term signals, the peak of DNA concentration will arrive earlier at the same position, and the peak will be more prominent than the whole curve. At a distance of 0.5mm from the cathode, we can reach the maximum communication rate of 0.909hz; at a distance of 4.5mm from the cathode, we can reach the maximum communication frequency of 0.2778hz, while at a distance of 9.5mm from the cathode, the maximum communication frequency is only 0.05348hz. The data shows that the communication system based on free diffusion is not suitable for long-distance communication.
  
  $\bullet$When the receiver is placed within the range of 5mm from the transmitter, it can achieve a faster transmission speed and strong anti-interference ability. After the range of 7mm, the signal interval must be increased to tens of seconds to suppress the inter code interference, and its ability to resist the inter code interference will also be reduced. 
  Considering the equilibrium constant, in the presence of OH-, the hydrolysis reaction of Zr4+ will occur. Because the reaction speed is proportional to the fourth power of COH -, the reaction speed is extremely slow at low OH - concentration. Under the observed time scale, it can be considered that the reaction will not occur without applying the excitation voltage. The voltage threshold of electrochemical reaction should be 2.17v. However, under the actual conditions, due to the influence of electrode impedance, solution conductivity and other factors, more voltage may be needed to start the reaction. By controlling the excitation voltage and duration, a DNA / Zr4 + monolayer can be released many times. Under the signal of excitation voltage of 5V and excitation duration of 1s, each layer of Zr4 + / DNA monolayer can be released six times at most.


\end{digest}


%TC:endignore

\end{document}
